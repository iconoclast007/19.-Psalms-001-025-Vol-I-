\section{Psalm 18 Comments}

\subsection{Numeric Nuggets}
Verses 5, 23, 27, 32, 36, and 45 use 13 words. Verses 4, 9, 18, 32, and 45 have 13 unique words. The word ``to'' is used 13 times in the chapter. The 13-letter word ``righteousness'' is found in the chapter.

\subsection{Psalm 18 Introduction}
This psalm is plainly messianic, and it's text parallels that of 2 Samuel 22, with slight variations. Consider this phenomena thoroughly. For example, there are numerous parallel accounts in Matthew, Mark, Luke, and John. Such instances are ripe fields for those who wish to read contradictions into scripture when (1) they do not start out with the assumptions (as we should) of inspiration and infallibility, (2) they (apparently) are not willing to figure out the resolution of these revelations,'' and (3) they cannot admit that they just do not know the answer. Ruckman points out that \cite{Ruckman1992PsalmsV1}:
\begin{compactenum}
    \item Two inspired accounts of the same event do not have to match word for word.
    \item The fact that two words in identical accounts do not match, doe snot mean there is a contradiction in either of them.
    \item A copyist can ``miscopy'' a word, and it still be inspired. Not only is the English text not word for word, but neither is any Massoretic text.
\end{compactenum}

\subsection{Psalm 18:2}
In verse 2, God is likened to seven things, and each of these has a characteristic which is true of Him \cite{Ruckman1992PsalmsV1}:
\begin{compactenum}
\item A Fortress is needed for defensive battle when one is besieged. 
\item A Deliverer is needed in the siege when food is running out.
\item Strength is needed to fight an aggressive warfare.
\item A Buckler is needed to hold your equipment together. 
\item An High Tower is needed for long distance observation.
\item The ``rock'' is the Rock of 1 Corinthians 10:4 and Deuteronomy 32:31. A Rock is needed for a stedfast, immovable foundation upon which to build the fortress. 
\item The seventh item is a ``horn,'' which is needed for mustering the troops to combat (Numbers 10:2), demoralizing the enemy (2 Chronicles 13:14), and celebrating the victory.
\end{compactenum}

\subsection{Psalm 18:4}
See the description of this ``king of terrors'' provided in Job 18:14,\footnote{\textbf{Job 18:14} - His confidence shall be rooted out of his tabernacle, and it shall bring him to the king of terrors.} and, hence, the connection with the book of Job. These ``ungodly men'' are described here, and in 2 Samuel 22:5\footnote{\textbf{2 Samuel 22:5} - When the waves of death compassed me, the floods of ungodly men made me afraid;}, 2 Peter 3:7\footnote{\textbf{2 Peter 3:7} - But the heavens and the earth, which are now, by the same word are kept in store, reserved unto fire against the day of judgment and perdition of ungodly men.}, and Jude 1:4.\footnote{\textbf{Jude 1:4} - For there are certain men crept in unawares, who were before of old ordained to this condemnation, ungodly men, turning the grace of our God into lasciviousness, and denying the only Lord God, and our Lord Jesus Christ.} These ungodly men are described with the attributes of a flood: (1) floods are common, (2) floods move with great speed and power, (3) Floods cannot be stopped, (4) floods have exceptional abilities to destroy lives and property, and (5) floods sweep away existing establishments, burying them. See references to ``a flood'' in scripture: Genesis 6:17, 9:11, 9>15, Job 22:16, Psalm 90:5, Isaiah 28:2, Isaiah 59:19, Jeremiah 46:7-8, Daniel 9:26, Amos 8:8, 9:5, and Revelation 12:5.

\subsection{Psalm 18:7}
Note the context in Revelation 6:12 and Revelation 11:19; it precedes the Advent. ``He bowed the heavens also, and came down'' literally. The Lord told Moses the day this would occur (Exod. 19:11, 18). And this time He will shake heaven and earth, according to Haggai 2:6 and Hebrews 12:25–26. ``He rode upon a cherub,'' so Ezekiel, chapter 1 is a picture of the Advent following the book immediately before it — Lamentations: a picture of the Jews in Uz in the land of Edom (Lam. 4:21). A storm and whirlwind accompany this appearance. ``Dark waters and thick clouds of the skies ... hail stones and coals of fire. The Lord also thundered in the heavens ... hail stones and coals of fire.'' The elements are listed in Job, chapter 37; Matthew 24:27, 29; Revelation 11:13; and Revelation 11:19. The hail stones are a repetition of Joshua 10:11, and that is why the Advent is likened to that battle in Habakkuk 3:7. \cite{Ruckman1992PsalmsV1}

\subsection{Psalm 18:20}
This cannot be the prayer of a Christian, because all of his righteousness is imputed. The same goes for verse 24.

\subsection{Psalm 18:50}
Here is another promotion: our Lord was singing praises to God ``in the midst of the church'' (Heb. 2:12), but now He is singing ``praises unto thy name'' among the heathen. Observe the subtle suggestion in verse 50 that there can be a King beside David. David is a type of ``the King,'' and so David’s seed is a type of Christ’s seed (see Ps. 22:30).\footnote{\textbf{Psalm 22:20} - Deliver my soul from the sword; my darling from the power of the dog.} In the Millennium, Jesus Christ is the King on the Throne of David (Luke 1:30–34), but David is a ``prince'' (Ezek. 44:3) among the people of Israel.\footnote{\textbf{Ezekiel 44:3} - It is for the prince; the prince, he shall sit in it to eat bread before the LORD; he shall enter by the way of the porch of that gate, and shall go out by the way of the same.}\footnote{\textbf{Luke 1:30-34} - And the angel said unto her, Fear not, Mary: for thou hast found favour with God. [31] And, behold, thou shalt conceive in thy womb, and bring forth a son, and shalt call his name JESUS. [32] He shall be great, and shall be called the Son of the Highest: and the Lord God shall give unto him the throne of his father David: [33] And he shall reign over the house of Jacob for ever; and of his kingdom there shall be no end. [34] Then said Mary unto the angel, How shall this be, seeing I know not a man?}\footnote{\textbf{Hebrews 2:12} - Saying, I will declare thy name unto my brethren, in the midst of the church will I sing praise unto thee.} \cite{Ruckman1992PsalmsV1}
