\subsection{Outlines from Ken Stanley}


%%%%%
\subsubsection{Hope In The Holy One}

\index[speaker]{Ken Stanley!Psalm 16:9 (Hope In The Holy One)}
\index[series]{Psalms (Ken Stanley)!Psalm 16:9 (Hope In The Holy One)}
\index[date]{2020/09/27!Psalm 16:9 (Hope In The Holy One) (Ken Stanley)}

\index[FACEBOOK]{ALLITERATED SERMON OUTLINES!Ken Stanley - Psalm 16:9!2020/09/27}

If we were to have this Psalm only we might find it hard to determine what and who David was talking about. We would wonder continuously about who the Holy One is. There are four references to the Holy One in Psalms:\\
\\
David here in Psalm 16:10 (Psalm 71:22, Psalm 78:41, Psalm 89:18\\
\\
Isaiah mentions Him 30 times\\
\\
Job, Jeremiah, Ezekiel, Daniel, Hosea, and Habakkuk would all make mention of the same Holy One. And, in nearly all of these OT references, the detail is further given that this is the Holy One of Israel! 
At the least, eleven different OT individuals had an concept of who this Holy One is. They didn’t know His name, didn’t know exactly who He was or when He would come; bu they knew He was real, they knew He was alive and they knew God would send Him! Then, in God’s divine providence and foreknowledge, He directed the Holy Spirit to inspire both Peter and Paul to make it abundantly clear whom all these were speaking of.\\
\\
Peter - Acts 2:25-31*\\
Paul - Acts 13:33-37*\\
\\
By having the blessing of the entire cannon of scripture, we can, without doubt and reservation, say that this Holy One is the Lord Jesus Christ!  The Lord Jesus Christ is our hope and we are it is because of this hope that we are awaiting the glorious appearing of the great God and our Saviour Jesus Christ...\\
\\
Psalm 16:9* - it is because of this same hope that we also can rest!\\
Hebrews 4:9 “There remained therefore a rest to the people of God.”\\
\\
There is an eternal rest that is coming and we will one day experience and enjoy that rest. And we will be able to do that because of what Jesus Christ, the Holy One has done! What does that rest mean? It’s not a rest of easing up, slowing down or quitting . It’s a rest that means to abide, dwell or reside. To settle down and establish. The hope we have in the Holy One is not a hope that makes us quit doing what we’re doing but to remain steadfast in where we’re going!\\

\begin{compactenum}[I.][8]
	\item We are \textbf{Not Disconnected}  \index[scripture]{Psalms!Psa 16:1-6}(Psa 16:1-6) There is a tether that connects us with our Heavenly Father! 
The Holy One has given us a hope that we are eternally connected to Him!
	\begin{compactenum}[A.][8]
	    \item The Preservation - vs. 1
 	    \item The Portion - vs. 5 
 	    \item The Pleasant Places - vs. 6
    \end{compactenum}
	\item We are \textbf{Not Damned}  \index[scripture]{Psalms!Psa 16:8}(Psa 16:8)
	\item We are \textbf{Not Doomed}  \index[scripture]{Psalms!Psa 16:10}(Psa 16:10)
	\item We are \textbf{Delighted}  \index[scripture]{Psalms!Psa 16:7}\index[scripture]{Psalms!Psa 16:11}(Psa 16:7, 11) The Holy One has not just kept us from some things but He has given us some things! 
Our hearts can be glad! We can glory and rejoice! And we can find rest in hope:
	\begin{compactenum}[A.][8]
	    \item Because of the Praise - vs. 7a
	    \item Because of the Prayer - vs. 7b
	    \item Because of the Path - 11a
	    \item Because of the Presence - 11b
	    \item Because of the Pleasure - 11c
    \end{compactenum}
	\item \textcolor{blue}{We are \textbf{Not Disappointed} [My addition]}
	\item \textcolor{blue}{We  \textbf{Will be Delivered} [My addition]}
\end{compactenum}


