\chapter{Psalm 25}
\footnote{\textcolor[rgb]{0.00,0.25,0.00}{\hyperlink{TOC}{Return to end of Table of Contents.}}}\textcolor[cmyk]{0.99998,1,0,0}{\emph{A Psalm} of David.}\\
\\
\textcolor[cmyk]{0.99998,1,0,0}{Unto thee, O LORD, do I lift up my soul.}
[2] \textcolor[cmyk]{0.99998,1,0,0}{O my God, I trust in thee: let me not be ashamed, let not mine enemies triumph over me.}
[3] \textcolor[cmyk]{0.99998,1,0,0}{Yea, let none that wait on thee be ashamed: let them be ashamed which transgress without cause.}
[4] \textcolor[cmyk]{0.99998,1,0,0}{Shew me thy ways, O LORD; teach me thy paths.}\footnote{Any saint’s soul should “wait” on God (vs. 3), trust in God (vs. 2), and look up to God (vs. 1). If the Lord shows you His “ways” and teaches you His “paths” (vs. 4), then you will rightly divide the word of truth (2 Tim. 2:15) and not be “ashamed” (vs. 3). No one can find 2 Timothy 2:15 in any English Bible any more unless he goes by the “King James Only,” to cite the Alexandrian Cult’s terminology, for God’s preservation of this unique truth is not found in the ASV, NASV, NIV, RSV, NRSV, RV, or NKJV. \cite{Ruckman1992Psalms}}
[5] \textcolor[cmyk]{0.99998,1,0,0}{Lead me in thy truth, and teach me: for thou \emph{art} the God of my salvation; on thee do I wait all the day.}
[6] \textcolor[cmyk]{0.99998,1,0,0}{Remember, O LORD, thy tender mercies and thy lovingkindnesses; for they \emph{have} \emph{been} ever of old.}\footnote{God’s “tender mercies” and “loving kindnesses” are manifest throughout the Scriptures and then are manifest in the lives of New Testament Christians for centuries. It is true that many of the saints taste the hellish circumstances of Hebrews 11:36--37, but these are the chosen “elect” for the martyr’s crown. Millions of Christians have come and gone off the face of this earth with no more sorrow and pain than that experienced by all of their unsaved neighbors around them, and these saints had the benefits of eternal security, the presence of Christ, the comfort of the Holy Spirit, and ``handfuls of purpose'' (Ruth 2:16) dumped out to them while they were going through the trials. \cite{Ruckman1992Psalms}}
[7] \textcolor[cmyk]{0.99998,1,0,0}{Remember not the sins of my youth, nor my transgressions: according to thy mercy remember thou me for thy goodness' sake, O LORD.}
[8] \textcolor[cmyk]{0.99998,1,0,0}{Good and upright \emph{is} the LORD: therefore will he teach sinners in the way.}
[9] \textcolor[cmyk]{0.99998,1,0,0}{The meek will he guide in judgment: and the meek will he teach his way.}
[10] \textcolor[cmyk]{0.99998,1,0,0}{All the paths of the LORD \emph{are} mercy and truth unto such as keep his covenant and his testimonies.}
[11] \textcolor[cmyk]{0.99998,1,0,0}{For thy name's sake, O LORD, pardon mine iniquity; for it \emph{is} great.}
[12] \textcolor[cmyk]{0.99998,1,0,0}{What man \emph{is} he that feareth the LORD? him shall he teach in the way \emph{that} he shall choose.}
[13] \textcolor[cmyk]{0.99998,1,0,0}{His soul shall dwell at ease; and his seed shall inherit the earth.}
[14] \textcolor[cmyk]{0.99998,1,0,0}{The secret of the LORD \emph{is} with them that fear him; and he will shew them his covenant.}
[15] \textcolor[cmyk]{0.99998,1,0,0}{Mine eyes \emph{are} ever toward the LORD; for he shall pluck my feet out of the net.}
[16] \textcolor[cmyk]{0.99998,1,0,0}{Turn thee unto me, and have mercy upon me; for I \emph{am} desolate and afflicted.}
[17] \textcolor[cmyk]{0.99998,1,0,0}{The troubles of my heart are enlarged: \emph{O} bring thou me out of my distresses.}\footnote{“Enlargement of the heart” is found in verse 17 as “fat” around the heart is found in Isa. 6:10. When surrounded by the adversities of this company—“troubles,” “pain,” “sins,” “hatred,” “desolate,” “afflicted,” “affliction,” and “the net” (vss. 15–19), any child of God should be flat on his face praying. The theme of the first forty-one Psalms—TRUST, is manifest throughout. Look at verses 1, 2, and 20. David says, “I wait on thee”; therefore, he is destined to “mount up with wings as eagles.” He will “run and not be weary” and will “walk and not faint” (Isa. 40:31). \cite{Ruckman1992Psalms}} \footnote{\textbf{Isaiah 6:10} -- Make the heart of this people fat, and make their ears heavy, and shut their eyes; lest they see with their eyes, and hear with their ears, and understand with their heart, and convert, and be healed.}
[18] \textcolor[cmyk]{0.99998,1,0,0}{Look upon mine affliction and my pain; and forgive all my sins.}
[19] \textcolor[cmyk]{0.99998,1,0,0}{Consider mine enemies; for they are many; and they hate me with cruel hatred.}
[20] \textcolor[cmyk]{0.99998,1,0,0}{O keep my soul, and deliver me: let me not be ashamed; for I put my trust in thee.}
[21] \textcolor[cmyk]{0.99998,1,0,0}{Let integrity and uprightness preserve me; for I wait on thee.}
[22] \textcolor[cmyk]{0.99998,1,0,0}{Redeem Israel, O God, out of all his troubles.}\footnote{In the closing verses of this Psalm, David makes another petition which is sixfold \cite{Ruckman1992Psalms}:
\begin{compactenum}
\item Turn unto me (vs. 16).
\item Look upon my affliction (vs. 18).
\item Consider my enemies (vs. 19).
\item Deliver me.
\item Bring me out of my distress (vs. 17).
\item Redeem me (vs. 22).
\item A seventh item could be: Preserve me (vs. 21).
\end{compactenum}}

