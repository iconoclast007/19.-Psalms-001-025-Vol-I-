\subsection{Outlines from Billheimer}

\subsubsection{The Grieving Shepherd}

\index[speaker]{Clarence Billheimer!Psalm 022 (The Grieving Shepherd)}
\index[series]{Psalms (Clarence Billheimer)!Psalm 022 (The Grieving Shepherd)}
\index[date]{2019/01/24!Psalm 022 (The Grieving Shepherd) (Clarence Billheimer)}
%%%%%
\index[FACEBOOK]{ALLITERATED SERMON OUTLINES!Clarence Billheimer (The Grieving Shepherd) - Psalm 22 (2019/01/24)}
%%%%%
\textbf{Introduction: } Other than Isaiah 53, there is no other passage in the Old Testament which pictures and prophesies the crucifixion of Christ, as well as Psalm 22, does.  Its presentation is remarkable when we compare it with the accounts in the gospels.  When people were crucified, their bodies were hung in such a way that it made breathing very difficult.  It’s not likely that Jesus said much more aloud than we read in the gospels, but He may have said many of the words of Psalm 22 in His mind and soul as He went through what you and I deserve.  I’ve often said that Jesus, as an infinite being, suffered in a finite period what a man, a finite being, would suffer in an infinite period.  Let us see the seven vocal sayings of Christ in this Psalm.\\

\begin{compactenum}[I.][8]
    \item Jesus was forsaken.
 The very words He uttered when God turned His back on His Son are the same words that begin this Psalm.  Christ was required to go through the same condition man without Christ will be in for eternity.
    \item Jesus was forgiving.
 He prayed for those who didn’t ask for it!  It reflects the model prayer, “Forgive us…as we forgive those…”
    \item Jesus was faithful.
 In His desire to honor His mother, He made sure there would be someone to care for her and help her through what she watched those moments.  We have no idea what went through her mind then.  The disciple Jesus loved was the only one besides her.  What a responsibility Jesus gave him!
    \item  Jesus was faced with thirst.
 “The water of life” sensed the same thirst as the man in hell who longed for one drop of water on the tip of his tongue.
    \item Jesus was fruitful.  Even in death, He was a soul winner.
    \item Jesus was finishing His work.
 “It is finished” is the most dynamic and powerful words He ever spoke.  It says nothing else needed to be done by Him or us!
    \item Jesus was finally with His Father again.
 He did not say, “I commit My spirit.”  He said, “I commend My spirit.”  He was expressing His desire for God to accept the sacrifice as a payment forever for sin.  In the book of Hebrews, we get another picture of what went on in heaven when He made that final statement.  True to His word, no man took His life.  He gave it willingly!\\
\end{compactenum}

\noindent \textbf{Conclusion:}  Think of the hymns we sing about this moment in history!  May we never grow weary or disrespectful of what Jesus did on the cross.  If we really want to know Him, we must get extremely well acquainted with what He was willing to do for us so we could get to know Him.



\subsubsection{The Singer’s Picture of the Suffering Saviour}

\index[speaker]{Clarence Billheimer!Psalm 022 (The Singer’s Picture of the Suffering Saviour)}
\index[series]{Psalms (Clarence Billheimer)!Psalm 022 (The Singer’s Picture of the Suffering Saviour)}
\index[date]{2020/01/19!Psalm 022 (The Singer’s Picture of the Suffering Saviour) (Clarence Billheimer)}
%%%%%
\index[FACEBOOK]{ALLITERATED SERMON OUTLINES!Clarence Billheimer (The Singer’s Picture of the Suffering Saviour) - Psalm 22 (2020/01/19)}
%%%%%
\textbf{Introduction: } How many songs can you immediately name with words about the cross of Christ? A more serious question: How many do you know have been written in the last generation or so? Most of the songs we love about the cross were not written recently. Today, if there is a good song written about the cross, it is one in a minority. One I love is, “He Loved Me with A Cross.” How sad, though that today one of the most important parts of our salvation is so neglected. Sadder still is how it is often trodden upon and disrespected. Some may call it a “slaughterhouse religion.” So many still have the image of Jesus on it (commonly called a crucifix). Some have dared to disgrace and discount the blood to the point of saying it absorbed into the ground below and has no main importance to salvation.\\
\\
Dr. Jack VanImpe used Psalm 22 and Isaiah 53 as key passages in his message, “The Greatest Love Story,” to show us how these two passages tell us more about the crucifixion of Christ than we may realize at first. In fact, there are things we read in the Old Testament we never read in the accounts of Matthew, Mark, Luke, or John. It truly is a dynamic and emotion-stirring message I have treasured through the years.\\
\\
Can you imagine David writing and singing this song, not realizing how prophetic it was? This one had to have been one of many he sang during his years of pursuit by Saul or perhaps Absalom. We cannot overlook this Psalm as one needed for the growing saint. Any saint who desires to grow as God’s child needs a healthy and awe-directed attitude toward His Saviour and what He went through so he could even be a saint!

\begin{compactenum}[I.][8]
    \item The forsaken shouter (22:1)
    \begin{compactenum}[A.][8]
        \item Jesus quoted it verbatim when God turned His back on Him.
        \item It speaks of the eternal separation of Man from God; the only place this can happen is hell.
    \end{compactenum}
    \item The forgotten sorrower (22:6-8)
    \begin{compactenum}[A.][8]
        \item In contrast to others of a different day or time
        \item The crowd said similar words about Jesus while they mocked Him on the cross.
    \end{compactenum}
    \item The forlorn sufferer (22:12-18)
    \begin{compactenum}[A.][8]
        \item Intimidated by apparent strength of those around
        \item Inundated with grief in the innermost body
        \item Incapable of any self-defense
    \end{compactenum}
    \item The future satisfaction (22:19-31)
    \begin{compactenum}[A.][8]
        \item Drawing from how God helped those in the past (22:4,5,9,24)
        \item Assured that this suffering is not the end of all hope
        \item The Saviour WILL get the final victory.
    \end{compactenum}
\end{compactenum}

\noindent  \textbf{Conclusion:} Reading this Psalm and linking David’s years of fear and peril with those Jesus went through in those hours on the cross helps us understand the hope David kept encouraging himself with. The Bible says that Jesus, “for the joy that was set before Him endured the cross, despising the shame.” As a man, Jesus did pray that God would allow the cup to pass from Him, and perhaps David may have prayed the same way. But Jesus always said, “Nevertheless, not My will, but Thine be done.” David could reflect on great victories God gave him, so every time he had the opportunity to execute HIS will against an enemy, he said, “Let God’s will be done with that person.”


\subsubsection{The Punishment of Christ}

\index[speaker]{Clarence Billheimer!Psalm 022 (The Punishment of Christ)}
\index[series]{Psalms (Clarence Billheimer)!Psalm 022 (The Punishment of Christ)}
\index[date]{2019/04/03!Psalm 022 (The Punishment of Christ) (Clarence Billheimer)}
%%%%%
\index[FACEBOOK]{ALLITERATED SERMON OUTLINES!Clarence Billheimer (The Punishment of Christ) - Psalm 22 (2019/04/03)}
%%%%%
\textbf{Introduction: } You will read more details about the punishment of Christ in the verses of Psalm 22 and Isaiah 53 than you do in Matthew, Mark, Luke, or John.  These men saw what He would endure hundreds of years before.  I wonder if they really comprehended it.  Dr. Jack VanImpe preached a powerful message, “The Greatest Love Story,” in which he spent considerable time with these passages and commented, “I am not a preacher who weeps easily, but when I read Isaiah 53, I wept.”  Well, if we are any kind of a Christian we should be and realize what it took to even make us worthy of heaven, we all ought to weep when this description is slowly and vividly presented.  I’ve seen it in some good movies.  I’ve seen it in some live presentations—toned down, of course, because no man could endure the exact same punishment Christ did.  In fact, many never made it past the whip, let alone the nailing to a cross.  As Dr. VanImpe further noted, “If an artist were to portray the crucifixion from these accounts in the Old Testament, no one would buy a copy.”  So, will you pray now that the Holy Spirit will burn afresh and anew in your mind and soul this price your Saviour paid for your sins?

\begin{compactenum}[I.][8]
    \item Christ suffered alone
    \begin{compactenum}[A.][8]
        \item All His followers forsook Him
        \item Only John and Mary (His mother) were at the cross
        \item God Himself forsook Him
    \end{compactenum}
    \item Christ suffered in agony
    \begin{compactenum}[A.][8]
        \item The beating with the whip
        \item The carrying of the cross
        \item The nails in His hands
        \item The dropping of the cross into the hole in the ground
    \end{compactenum}
    \item Christ made an eternal atonement
    \begin{compactenum}[A.][8]
        \item For the payment of sin
        \item For the pardon of sin
        \item For the punishment of the sinner
    \end{compactenum}
    \item Christ’s death accomplished God’s eternal plan
    \begin{compactenum}[A.][8]
        \item As had been seen in animal sacrifices
        \item As had been pictured in the Tabernacle
        \item When He said, “IT IS FINISHED!”
    \end{compactenum}
\end{compactenum}

\noindent  \textbf{Conclusion:} Christ’s death on the cross was not just an act of martyrdom.  Christ’s death on the cross was far from and much greater than the common punishment given criminals in those days.  Christ died for Himself as well as for others.  He died for Himself as a picture of total submission to the eternal plan of God they had worked out ages ago.  He died for others because there was no other way for reconciliation to God.  He died exclusively for man—not angels, not some other terrestrial being that may exist on another planet.  Christ died for YOU—He would have done this if you had been the only individual who needed it done for them!  All He asks now is for us to accept that payment for our sins by faith.


