\section{Psalm 25 Comments}

\subsection{Numeric Nuggets}
Verses Psalm 24:11 and 24:13 have 13 words.

%%% Ruckman notes
%%  This is by far the simplest Psalm we have had to deal with yet. It contains more purely devotional material than any one of the preceding twenty-four. Yet even in this one, we find an ending inserted that says “Redeem Israel, O God, out of all his troubles,” which lands us right back into Daniel’s Seventieth Week again. But there are twentyone verses that any saint in any dispensation could profitably apply to his own condition and circumstances. Any saint’s soul should “wait” on God (vs. 3), trust in God (vs. 2), and look up to God (vs. 1). If the Lord shows you His “ways” and teaches you His “paths” (vs. 4), then you will rightly divide the word of truth (2 Tim. 2:15) and not be “ashamed” (vs. 3). No one can find 2 Timothy 2:15 in any English Bible any more unless he goes by the “King James Only,” to cite the Alexandrian Cult’s terminology, for God’s preservation of this unique truth is not found in the ASV, NASV, NIV, RSV, NRSV, RV, or NKJV. “Let not mine enemies triumph over me” is a legitimate prayer for any saint endangered by enemies, although the “grounds” for getting this prayer answered differ as much as David’s case differs from Paul’s. “Them...which transgress without cause” is very interesting for it shows that some folks have a CAUSE or REASON for transgressing even if God doesn’t accept it. Whosoever is angry with his brother “without a cause” (Matt. 5:22) is in danger, but Paul says that if there is a CAUSE, “Be ye angry” (Eph. 4:26). The transgression here means a trespass against the speaker, when the speaker (David) has done nothing to warrant the trespass. An exact case is Saul when he hunts him down to kill him. There are four requests in the prayer: 1. SHEW ME (vs. 4). 2. TEACH ME (vs. 4). 3. LEAD ME (vs. 5). 4. REMEMBER ME (vs. 7). In line with the last request, the Lord “remembered Noah” (Gen. 8:1), but not for his drinking. Noah is pictured in Hebrews 11:7 without fault. The Lord “remembered Rachel” (Gen. 30:22), but did not “remember” the “sins of her youth.” He doesn’t mention Rachel stealing her daddy’s “gods” in the New Testament (Matt. 2:18). God’s “tender mercies” and “loving kindnesses” are manifest throughout the Scriptures and then are manifest in the lives of New Testament Christians for centuries. It is true that many of the saints taste the hellish circumstances of Hebrews 11:36–37, but these are the chosen “elect” for the martyr’s crown. Millions of Christians have come and gone off the face of this earth with no moresorrow and pain than that experienced by all of their unsaved neighbors around them, and these saints had the benefits of eternal security, the presence of Christ, the comfort of the Holy Spirit, and “handfuls of purpose” (Ruth 2:16) dumped out to them while they were going through the trials. 

%% There is nothing “Messianic” about verses 7 and 11. This is David. He is praying for forgiveness of sin in verse 18. God will teach because He is good (vs. 8), but it must be “in the way” (vs. 8) because He is “the way, the truth, and the life” (John 14:6). This means that “the way” must be “his way” (vs. 9). Notice how Rebekah, a type of the Bride of Christ, has to go “his way” (Gen. 24:61)— Eliezer’s way, who is a type of the Holy Spirit. The instruction is promised to only one kind of a Christian: a meek Christian (vs. 9). This is one of the fruits of the Holy Spirit found in Galatians 5:23; it has to do with the heart attitude of the saint in relation to God— NOT MAN. Observe that the man who was “meek above all the men” upon the earth (Num. 12:3) was a KILLER (Ex. 2:12), who lost his temper on more than one occasion (Ex. 32:19; Num. 20:10–11). You see, these modern, humanistic liberals have been teaching Christians a lie, and they have repeated their little “turn the cheek” bit so often that modern Christians think that a Christian is a milk sop who lies down flat on his face every time a jackass walks into the house, or who kisses and hugs every Biblerejecting, Christ-denying, God-hating hellcat on earth. That is not what the word “meek” means in the book. Observe, in verse 9, that you cannot “guide” a bucking bronco or a stubborn ass anywhere. When a horse got his rear foot caught in the stirrup while trying to scratch himself, his rider said, “Listen bud, they’re ain’t room for two of us up here, and if you’re going to drive, I’m gettin’ off!” “His covenant and his testimonies” land us back on Israel again in the works and faith setup of the Old Testament, but still there is spiritual truth if you want to talk about a Christian “living for the Lord.” All the paths of the Lord are TRUTH even when the path is dark and thorny, or going uphill. None of the Lord’s paths are crooked (see Prov. 2:13, 15, 20). The prayers of David for forgiveness (see vss. 11, 18) are a “no-no” to the Dry Cleaners (Stam, Baker, O’Hair, Bullinger, Ballard, Watkins, Moore, Brock, Sharpe, et al.), for these Antinomians who are wrongly dividing the word of truth assume that you “make the cross of none effect” or become an “enemy of the cross of Christ” if you ask for forgiveness AFTER salvation because Calvary “covers it all.” The dry cleaned “idjits” don’t realize that sin can break your fellowship with the Lord, no matter if ALL of them were paid for eternally. Their thinking is that you can get back into fellowship with the Lord without apologizing for lying to the Holy Spirit (Acts 5:3), grieving the Holy Spirit (Eph. 4:30), and quenching the Holy Spirit (1 Thess. 5:19). The hyper-dispensational “grace position” (Bullinger, Baker, and Stam were all five point TULIP hyper-Calvinists) is that a son can stay in fellowship with a father after tearing the curtains down, cutting up the rug, burning the bed sheets, and dumping ink on the pillow cases, by saying, “Thank you for taking care of this at Calvary.” It doesn’t work that way. Christians who take that “grace” position are WEAK spiritually and SICK physically, and many are DEAD (1 Cor. 11:30). If you want to know why no dry cleaning “Hyper” ever did one really spiritual work for God in a century it is because God will not honor this attitude about sin. He dumps them.

