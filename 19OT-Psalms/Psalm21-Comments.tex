\section{Psalm 21 Comments}

\subsection{Numeric Nuggets}
\textbf{18:} There are 13 verses in the psalm. Verses 1, 4, 6, 8, and 11 have 18 words. 

\subsection{Psalm 21 Introduction}
The psalm switches back and forth between David's kingship and Christ's kingship.

\subsection{Psalm 21:3}
Read the word ``preventest'' as ``pre-eventest'' - something that happens before something else. Here, the bestowing of goodness came first and then the crown.  Forms of the word ``prevent''a re found 17 times in the, AV, with the most famous probably being 1 Thessalonians 4:15, which speaks of the order of events at the Rapture of the Church.\footnote{\textbf{1 Thessalonians 4:15} - For this we say unto you by the word of the Lord, that we which are alive and remain unto the coming of the Lord shall not prevent them which are asleep.}

\subsection{Psalm 21:11}
The word ``mischievous'' is used 5 times in scripture: (1) Psalms 21:11, (2) Psalms 38:12, (3) Proverbs 24:8, (4) Ecclesiastes 10:13, and (5) Micah 7:3. The sources is identified in Acts 13:10.

%% Ruckman Notes
%% The whole Psalm is about the Lord Jesus Christ, most of it again dealing with His Second Coming and coronation as “King over all the earth” (Zech. 14:9; Ps. 47:2;  Ps.  72). As usual, David switches back and forth from his own kingship to Christ’s kingship. “The king” in verse 1 is David rejoicing in the Lord. He is trusting in the Lord in verse 7 (the theme of the first forty-one Psalms).  But there are overtones in verses 2, 3, and 4 that take us far beyond David. Now, we have spoken at length on verse 4 in The Commentary on the Pastoral Epistle (see Titus 1:1–2). The long and short of it is that in making an ETERNAL payment for sin (which required ETERNAL DEATH) the Lord Jesus Christ was promised ETERNAL LIFE—before Genesis 1:1. David’s “heart’s  desire” may have been to have a son on his throne, but when he is given the “sure promises” of an everlasting seed and throne (2 Sam. 7:16), it comes to him as a complete surprise (2 Sam. 7:18); evidently he had NOT asked for what was given. The words of Psalm 21 are literally true of one man, and one man only. Look at them: 1. He has eternal life (vs. 4). 2. His glory is great (vs. 5). 3. He has honour and majesty (vs. 5) 4. He is blessed forever (vs. 6) 5. God will destroy ALL of His enemies (vss. 8–10). David is speaking of “the King of the Jews” (John 19:19). So in the next Psalm he describes what took place when THAT inscription was placed over the King’s head. Devotionally, we may ask, “Do you ‘greatly rejoice,’ as David, who danced although he was a king?’’ (vs. 1) You need goodness (vs. 3) before you need wealth or health; goodness is a “blessing.” Imagine GOD crowning a MAN (vs. 3) (the only time this ever happened was when Pope Leo crowned Charlemagne, Christmas Eve, A.D. 800, and claimed that “God was doing the crowning.” Heil papa! Vatican über alles!). Note, in verse 4, that eternal life can be had for the ASKING; did you get that? Get it out of Romans 10:13 if you didn’t get it here. If God “ONLY hath immortality” (see 1 Tim. 6:16) then He can give it to sinners (1 Cor. 15:49– 55). “Glory...honour, and majesty” (vs. 5) are either fleshy or demoniac when they are not “sanctified.” Salvation sanctifies “glory...honour, and majesty.” Without salvation (“education without salvation is damnation”) glory, honour, and majesty turn out like they turned out when connected with Caesar Augustus, bloody Mary, Adolph Hitler, Pope Leo the Great, M. L. King Jr., Mandela, Castro, King Henry VIII, Napoleon, Tully, Wallenstein, Marx, Freud, Einstein, FDR, and Pope Paul VI: disruption, demolition, and catastrophe. Alexander  the Great had all three; so did Napoleon and Abraham Lincoln.
%%
%% Verse 9 matches Malachi 4:1–4 and 2 Thessalonians 2:8. John the Baptist makes reference to it in Matthew 3:10, 12. Verse 10 is commented on by the Holy Spirit in Isaiah 14:20, and has to do with the “seed” of the Son of Perdition (cf. also Job 18:19). “They intended evil against thee.” Like Joseph’s brothers (Gen. 50:20) they intended to annihilate the “heir” (see Matt. 21:38). “They imagined a mischievous device” according to Acts 3:14–15 and Acts 2:22–24, but they were “not able to perform” it because the “victim” didn’t stay dead. (“You can’t keep a good man down!”). He rose from the dead (see vs. 4). They planned to run the world without Him. “Evil,” in this case, means: The CFR, NAACP, NEA, AMA, CIA, FBI, HEW, NCCC, UN, League of Nations, Triple Entente, Holy Alliance, NATO Treaty, Warsaw Pact, planned economy, birth control, air conditioning, computers, Vatican politics, the Bilderbergers, Illuminati, Federal Reserve System, ASV, Democratic Party platform, urban renewal, the NASV, affirmative action, revenue sharing, agrarian reforms, the NIV, and “city hall.” All scientific, social, economic, educational, political, and religious “progress” is simply “man” trying to get rid of God (see The Commentary on Genesis: Gen. 3:16–19 and Gen. 9:1–3). They want to run the vineyard (Matt. 21:38). They will not have “this man” to reign over them (Luke 19:14). He will, anyway, for He will “make them turn their back” when He makes ready his arrows “against the face of them” (vs. 12). Jesus Christ will be exalted in His own strength (vs. 13), and then we, along with Israel, will “sing and praise” His power (vs. 13). If He could be exalted only through our strength, God would get no real praise.
