\subsection{Outlines from Others}

\subsubsection{Eight Words to Pray}
\index[speaker]{Jim Irwin!Psalm 020 (Eight Words to Pray)}
\index[series]{Psalms (Jim Irwin)!Psalm 020 (Eight Words to Pray)}
\index[date]{2016/04/16!Psalm 020 (Eight Words to Pray) (Jim Irwin)}
\textbf{Introduction: }Living next to the Gulf Coast has sensitized me to “the calm before the storm,” that eerie moment of silence just before the winds and rains crash in upon us. The skies become leaden. The wind subsides momentarily. The smell of rain is in the air. It is as if nature pauses before its holocaust breaks loose. Similarly, life has its moments of calm. Battlefields lie quiet; then the bombardment begins. Anxious reporters freeze as news of the president’s condition comes from the emergency room. Marital strain can grip a family in silence before the cracks appear. Psalm 20 is such a pause. Israel is ready for battle; the “day of trouble” has come. The legions with their banners are ordered for war. But while pagans trust in chariots and horses, God’s people trust in His name. In “the calm before the storm,” the commanders go up to the temple with their troops where the king offers his sacrifice and Israel is blessed for battle. Only when spiritual preparation is completed can the opposing forces be joined. Many people want to have victory in life. They want to see success in everything they do. Here, David prays for victory in the oncoming battle. He asks for God to hand him victory. He admits that other people trust in other things to gain victory. David only trusts God. But just because he doesn’t trust in other ways for success, that doesn’t prevent him from making the “big ask.” Eight times, David claims that God can do something for him to provide him victory. David prayed to God for victory in his circumstances. God helped him. David was in a very tight spot. But God helped him. Just as God helped David, He can also help you. I agree with John Calvin about this psalm. He said: Many interpreters view this prayer as offered up only on one particular occasion; but in this I cannot agree. The occasion of its composition at first may have arisen from some particular battle which was about to be fought, either against the Ammonites, or against some other enemies of Israel. But the design of the Holy Spirit, in my judgment, was to deliver to the Church a common form of prayer, which, as we may gather from the words, was to be used whenever she was threatened with any danger. These requests were from a king who was ready for battle against a national foe. I believe that we can personalize these requests from a child of a king who is ready for battle against a spiritual foe. So I want us to look at these prayers as petitions we can ask from God in our own lives. \href{http://www.patheos.com/blogs/jimerwin/2016/04/18/psalm-201-9-trusting-god-prayer/\#Mgi8KGIip8tAbGeA.99}{Read more.}
\begin{compactenum}[I.][8]
    \item \textbf{Answer} Me \index[scripture]{Psalms!Psa 020:01}\index[scripture]{Psalms!Psa 020:09}(Psalm 20:1,9)
    \item \textbf{Protect} Me \index[scripture]{Psalms!Psa 020:01}(Psalm 20:1)
    \item \textbf{Help} Me \index[scripture]{Psalms!Psa 020:02}(Psalm 20:2)
    \item \textbf{Sustain} Me \index[scripture]{Psalms!Psa 020:02}(Psalm 20:2)
    \item \textbf{Remember} Me \index[scripture]{Psalms!Psa 020:03}(Psalm 20:3)
    \item \textbf{Give} Me \index[scripture]{Psalms!Psa 020:04}(Psalm 20:4)
    \item \textbf{Fulfill} Me \index[scripture]{Psalms!Psa 020:04}(Psalm 20:4)
    \item \textbf{Lift} Me \index[scripture]{Psalms!Psa 020:05--08}(Psalm 20:5--8)
\end{compactenum} 
