\section{Psalm 15 Comments}

\subsection{Numeric Nuggets}
\textbf{5:} and \textbf{6:} Psalm 15 use the word ``his'' five times and adds the italic ``\emph{his}'' another time. In both cases, un-redeemed man (6 is the number of man) is destined for death (5 is the number of death).\\
\\
\textbf{13:} Verse 2 contains 13 unique words. The 13-letter word ``righteousness'' is found in the chapter. The word ``holy'' is the 13$^{th}$ word in the chapter.

\subsection{Psalm 15:1}
The fundamental problem the self-examining man must answer is verse 1. How do I get to God, or heaven? Who gets to go? Why? How? The answer, or answers, are found in scripture.  It is insufficient, arrogant, and vanity to come up with our own answers. We must seek out what God says about the matter. And just what if there are multiple, different, and even contradictory answers? Do you think it just might be wise to sort these out? Pretending there are no differences must just be a unwise approach, especially since getting this question wrong will land some body in Hell. Perhaps we ought to find out what God requires, and then meet that requirement. Evidently, from Psalm 15, someone, at sometime, has a different set of rules than I do, saved by grace through faith.\\
\\
Let's start out by pointing Ephesians 2:8-9, for the New Testament Christian. I am headed to heaven simply by belief. I am very likely guilt of things listed in Psalm 15, but I have a substitute, a replacement, who is not. I have admitted my guilt, acknowledged by deserved punishment, and accepted the payment of Jesus Christ on my behalf. Afterwards, I act as the new creature I have been made. The works come afterward. Here, in Psalm 15, the words come before. \footnote{\textbf{Ephesians 2:8-9} - For by grace are ye saved through faith; and that not of yourselves: it is the gift of God: [9] Not of works, lest any man should boast.}\\
\\
This psalm, though, presents a list of things one must do to get to the ``tabernacle'' and to the ``holy hill.'' The list:
\begin{compactenum}
    \item Walk uprightly [1],
    \item Work righteousness [1],
    \item Speaks the truth in his heart [1],
    \item Does not backbite [2],
    \item Does not do evil to his neighbour [2],
    \item Does not take up a reproach against his neighbour [2],
    \item Recognizes and condemns evil [3],
    \item Honours them that fear the Lord [3],
    \item Swears to his own hurt [3],
    \item Doesn't change [3] (endures),
    \item Does not loan with interest [4], and
    \item Does not take bribes [4].\\
\end{compactenum}
\noindent The same question is asked in Psalm 24:3.\footnote{\textbf{Psalm 24:3} - Who shall ascend into the hill of the LORD? or who shall stand in his holy place?} Psalm 24 is the parallel passage. In it, as here, the Lord is reigning on earth, and has a tabernacle, on a holy hill, as described in Psalm 2:6.\footnote{\textbf{Psalm 2:6} - Yet have I set my king upon my holy hill of Zion}. This is after God has spoken to the heathen in wrath (Psalm 2:5) and vexed them. The Lord is not currently reigning on earth, not in Jerusalem, or on any of the seven hills of Rome. The context is beautifully described in Isaiah 2:2-5 and 33:20-24.\footnote{\textbf{Isaiah 2:2-5} - And it shall come to pass in the last days, that the mountain of the LORD’S house shall be established in the top of the mountains, and shall be exalted above the hills; and all nations shall flow unto it. [3] And many people shall go and say, Come ye, and let us go up to the mountain of the LORD, to the house of the God of Jacob; and he will teach us of his ways, and we will walk in his paths: for out of Zion shall go forth the law, and the word of the LORD from Jerusalem. [4] And he shall judge among the nations, and shall rebuke many people: and they shall beat their swords into plowshares, and their spears into pruninghooks: nation shall not lift up sword against nation, neither shall they learn war any more. [5] O house of Jacob, come ye, and let us walk in the light of the LORD.}\footnote{\textbf{Isaiah 33:20-24} - Look upon Zion, the city of our solemnities: thine eyes shall see Jerusalem a quiet habitation, a tabernacle that shall not be taken down; not one of the stakes thereof shall ever be removed, neither shall any of the cords thereof be broken. [21] But there the glorious LORD will be unto us a place of broad rivers and streams; wherein shall go no galley with oars, neither shall gallant ship pass thereby. [22] For the LORD is our judge, the LORD is our lawgiver, the LORD is our king; he will save us. [23] Thy tacklings are loosed; they could not well strengthen their mast, they could not spread the sail: then is the prey of a great spoil divided; the lame take the prey. [24] And the inhabitant shall not say, I am sick: the people that dwell therein shall be forgiven their iniquity.}This temple is described in Ezekiel 4-46. It is a Jewish context. The Lord is on a throne in Jerusalem in the midst of a Jewish Millennial kingdom, a kingdom of righteousness. Remember John 4:22? Salvation is of the Jew.\footnote{\textbf{John 4:22} - \textcolor[cmyk]{0,1,0,0}{Ye worship ye know not what: we know what we worship: for salvation is of the Jews.}}\\
\\
Psalm 15 is describing the works necessary, the conditions, necessary to enter this rebuilt temple. Consider the warning passages, pointed out by Ruckman: Psalm 18:20\footnote{\textbf{Psalm 18:20} - The LORD rewarded me according to my righteousness; according to the cleanness of my hands hath he recompensed me.}, Matthew 24:13\footnote{\textbf{Matthew 24:13} - \textcolor[cmyk]{0,1,0,0}{But he that shall endure unto the end, the same shall be saved.}}, Hebrews 3:14\footnote{\textbf{Hebrews 3:14} - For we are made partakers of Christ, if we hold the beginning of our confidence stedfast unto the end;}, 6:6\footnote{\textbf{Hebrews 6:6} -  If they shall fall away, to renew them again unto repentance; seeing they crucify to themselves the Son of God afresh, and put him to an open shame.}, 10:23\footnote{\textbf{Hebrews 10:23} -  If they shall fall away, to renew them again unto repentance; seeing they crucify to themselves the Son of God afresh, and put him to an open shame.}, Revelation 12:11\footnote{\textbf{Revelation 12:11} - And they overcame him by the blood of the Lamb, and by the word of their testimony; and they loved not their lives unto the death.}, 14:4 \footnote{\textbf{Revelation 14:4} - These are they which were not defiled with women; for they are virgins. These are they which follow the Lamb whithersoever he goeth. These were redeemed from among men, being the firstfruits unto God and to the Lamb.}. \cite{Ruckman1992PsalmsV1} \\
\\
Kidner\cite{kidner2014psalmsV1} and Phillips have the psalm describing a ``guest in the Lord's house,'' and focusing on expected and proper behavior once inside. This is different, of course, than actual requirements to get in! Phillips does, though, describe the probable connection between Psalm 15 and the Sermon on the Mount.  \cite[121]{Phillips2001ExploringPsalms1}\\
\\
\noindent Boice comments on Psalm 15 by pointing out that David's answers were \emph{representative} - his list, here, was not all-inclusive. Other lists are provided in Psalm 24:3-4\footnote{\textbf{Psalm 24:3-4} - Who shall ascend into the hill of the LORD? or who shall stand in his holy place? [4] He that hath clean hands, and a pure heart; who hath not lifted up his soul unto vanity, nor sworn deceitfully.} and Isaiah 33:14-17.\footnote{\textbf{Isaiah 33:14-17} - The sinners in Zion are afraid; fearfulness hath surprised the hypocrites. Who among us shall dwell with the devouring fire? who among us shall dwell with everlasting burnings? 15 [He] that walketh righteously, and speaketh uprightly; he that despiseth the gain of oppressions, that shaketh his hands from holding of bribes, that stoppeth his ears from hearing of blood, and shutteth his eyes from seeing evil; [16] He shall dwell on high: his place of defence shall be the munitions of rocks: bread shall be given him; his waters shall be sure. [17] Thine eyes shall see the king in his beauty: they shall behold the land that is very far off.}\cite{boice2005psalmsV1} He continues with comments on the six couplets he sees in the psalm, which follow the parallelism in the Hebrew text. These couplets provide an outline for the psalm, ``The Person God Approves'':\\

\begin{compactenum}[I.]
    \item His character 
    \item His speech
    \item His conduct
    \item His values
    \item His integrity
    \item His use of money\\

\end{compactenum}

\noindent Psalm 15 is captured in the hymn by Isaac Watts:
\begin{quote}
Who shall ascend thy heav'nly place,\\
Great God, and dwell before thy face?\\
The man that minds religion now,\\
And humbly walks with God below;\\
\\
Whose hands are pure, whose heart is clean,\\
Whose lips still speak the thing they mean;\\
No slanders dwell upon his tongue;\\
He hates to do his neighbor wrong.\\
\\
- Scarce will he trust an ill report,\\
Nor vents it to his neighbor's hurt:\\
Sinners of state he can despise,\\
But saints are honored in his eyes. -\\
\\
- Firm to his word he ever stood,\\
And always makes his promise good;\\
Nor dares to change the thing he swears,\\
Whatever pain or loss he bears.- \\
\\
- He never deals in bribing gold,\\
And mourns that justice should be sold;\\
While others gripe and grind the poor,\\
Sweet charity attends his door. -\\
\\
He loves his enemies, and prays\\
For those that curse him to his face\\
And doth to all men still the same\\
That he would hope or wish from them.\\
\\
Yet, when his holiest works are done,\\
His soul depends on grace alone:\\
This is the man thy face shall see,\\
And dwell for ever, Lord, with thee.   \\
\end{quote}