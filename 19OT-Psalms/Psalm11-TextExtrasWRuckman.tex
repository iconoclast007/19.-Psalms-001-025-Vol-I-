\chapter{Psalm 11}
\footnote{\textcolor[rgb]{0.00,0.25,0.00}{\hyperlink{TOC}{Return to end of Table of Contents.}}}\textcolor[rgb]{0.00,0.00,1.00}{To the chief Musician, \emph{A Psalm} of David.}\\
\\
\textcolor[rgb]{0.00,0.00,1.00}{In the LORD put I my trust: how say ye to my soul, Flee \emph{as} a bird to your mountain?}\index[AWIP]{In!Psalms!Psa 011:001}\index[AWIP]{the!Psalms!Psa 011:001}\index[AWIP]{LORD!Psalms!Psa 011:001}\index[AWIP]{put!Psalms!Psa 011:001}\index[AWIP]{I!Psalms!Psa 011:001}\index[AWIP]{my!Psalms!Psa 011:001}\index[AWIP]{trust!Psalms!Psa 011:001}\index[AWIP]{how!Psalms!Psa 011:001}\index[AWIP]{say!Psalms!Psa 011:001}\index[AWIP]{ye!Psalms!Psa 011:001}\index[AWIP]{to!Psalms!Psa 011:001}\index[AWIP]{my!Psalms!Psa 011:001 (2)}\index[AWIP]{soul!Psalms!Psa 011:001}\index[AWIP]{Flee!Psalms!Psa 011:001}\index[AWIP]{\emph{as}!Psalms!Psa 011:001}\index[AWIP]{a!Psalms!Psa 011:001}\index[AWIP]{bird!Psalms!Psa 011:001}\index[AWIP]{to!Psalms!Psa 011:001 (2)}\index[AWIP]{your!Psalms!Psa 011:001}\index[AWIP]{mountain!Psalms!Psa 011:001}\index[NWIV]{20!Psalms!Psa 011:001}\index[PNIP]{I!Psalms!Psa 011:001}\index[PNIP]{LORD!Psalms!Psa 011:001}
[2] \textcolor[rgb]{0.00,0.00,1.00}{For, lo, the wicked bend \emph{their} bow, they make ready their arrow upon the string, that they may privily shoot at the upright in heart.}\index[AWIP]{For!Psalms!Psa 011:002}\index[AWIP]{lo!Psalms!Psa 011:002}\index[AWIP]{the!Psalms!Psa 011:002}\index[AWIP]{wicked!Psalms!Psa 011:002}\index[AWIP]{bend!Psalms!Psa 011:002}\index[AWIP]{\emph{their}!Psalms!Psa 011:002}\index[AWIP]{bow!Psalms!Psa 011:002}\index[AWIP]{they!Psalms!Psa 011:002}\index[AWIP]{make!Psalms!Psa 011:002}\index[AWIP]{ready!Psalms!Psa 011:002}\index[AWIP]{their!Psalms!Psa 011:002}\index[AWIP]{arrow!Psalms!Psa 011:002}\index[AWIP]{upon!Psalms!Psa 011:002}\index[AWIP]{the!Psalms!Psa 011:002 (2)}\index[AWIP]{string!Psalms!Psa 011:002}\index[AWIP]{that!Psalms!Psa 011:002}\index[AWIP]{they!Psalms!Psa 011:002 (2)}\index[AWIP]{may!Psalms!Psa 011:002}\index[AWIP]{privily!Psalms!Psa 011:002}\index[AWIP]{shoot!Psalms!Psa 011:002}\index[AWIP]{at!Psalms!Psa 011:002}\index[AWIP]{the!Psalms!Psa 011:002 (3)}\index[AWIP]{upright!Psalms!Psa 011:002}\index[AWIP]{in!Psalms!Psa 011:002}\index[AWIP]{heart!Psalms!Psa 011:002}\index[NWIV]{25!Psalms!Psa 011:002}\footnote{The first seven words in verse one are self explanatory; they emphasize the theme of the first forty-one Psalms: trusting the Lord (see Ps. 118:8 and Jer. 17:5 for the Holy Spirit’s comments, independent of all the Hebrew and Greek scholars and translators).}\footnote{“Flee as a bird to your mountain?” “If I am trusting in the Lord, what is He doing telling me to run for my life?” Easy: the Psalmist is in Matthew 24:16, so he will need to fly like a dove (Ps. 55:6) and pray that his “flight” is not “on the sabbath” (Matt. 24:20). God did not bear Israel on “eagles’ wings” (Exod. 19:4) coming out of Egypt into the wilderness without a reason; it will happen again. Spiritually speaking, we may ask ourselves: “Why flee to a mountain when we can trust in the One who made the mountains?” “The wicked” (vs. 2) is right on the spot, as he was in Psalm 9:17 and 10:2–4. Verse 2 is both literal and figurative (see Joseph in Gen. 49, for the figure).}
[3] \textcolor[rgb]{0.00,0.00,1.00}{If the foundations be destroyed, what can the righteous do?}\index[AWIP]{If!Psalms!Psa 011:003}\index[AWIP]{the!Psalms!Psa 011:003}\index[AWIP]{foundations!Psalms!Psa 011:003}\index[AWIP]{be!Psalms!Psa 011:003}\index[AWIP]{destroyed!Psalms!Psa 011:003}\index[AWIP]{what!Psalms!Psa 011:003}\index[AWIP]{can!Psalms!Psa 011:003}\index[AWIP]{the!Psalms!Psa 011:003 (2)}\index[AWIP]{righteous!Psalms!Psa 011:003}\index[AWIP]{do!Psalms!Psa 011:003}\index[NWIV]{10!Psalms!Psa 011:003}\footnote{Well, they will have to “flee like a bird” with the “wings of a dove” and pray their flight “be not on the Sabbath.” When the Son of Perdition shows up, the foundations of law and order are gone, the foundations of Biblical Christianity are gone, the foundations of economics and liberty are gone, and the foundations of the Temple (Rev. 11; Ps. 137:7) and Judaism are gone.}
[4] \textcolor[rgb]{0.00,0.00,1.00}{The LORD \emph{is} in his holy temple, the LORD'S throne \emph{is} in heaven: his eyes behold, his eyelids try, the children of men.}\index[AWIP]{The!Psalms!Psa 011:004}\index[AWIP]{LORD!Psalms!Psa 011:004}\index[AWIP]{\emph{is}!Psalms!Psa 011:004}\index[AWIP]{in!Psalms!Psa 011:004}\index[AWIP]{his!Psalms!Psa 011:004}\index[AWIP]{holy!Psalms!Psa 011:004}\index[AWIP]{temple!Psalms!Psa 011:004}\index[AWIP]{the!Psalms!Psa 011:004}\index[AWIP]{LORD'S!Psalms!Psa 011:004}\index[AWIP]{throne!Psalms!Psa 011:004}\index[AWIP]{\emph{is}!Psalms!Psa 011:004 (2)}\index[AWIP]{in!Psalms!Psa 011:004 (2)}\index[AWIP]{heaven!Psalms!Psa 011:004}\index[AWIP]{his!Psalms!Psa 011:004 (2)}\index[AWIP]{eyes!Psalms!Psa 011:004}\index[AWIP]{behold!Psalms!Psa 011:004}\index[AWIP]{his!Psalms!Psa 011:004 (3)}\index[AWIP]{eyelids!Psalms!Psa 011:004}\index[AWIP]{try!Psalms!Psa 011:004}\index[AWIP]{the!Psalms!Psa 011:004 (2)}\index[AWIP]{children!Psalms!Psa 011:004}\index[AWIP]{of!Psalms!Psa 011:004}\index[AWIP]{men!Psalms!Psa 011:004}\index[NWIV]{23!Psalms!Psa 011:004}\index[PNIP]{LORD!Psalms!Psa 011:004}\index[PNIP]{LORD'S!Psalms!Psa 011:004}\footnote{The ``holy temple'' is like the one in Habakkuk. “His eyelids try the children of men.” This is what they call an “anthropomorphism,” which is a ten dollar way of saying that human attributes are ascribed to God so a human can understand what God is trying to tell him. The Holy Spirit’s comments are found in Proverbs 15:3, 20:8, and Psalm 66:7. “The Lord trieth the righteous.” A trial is not always a punishment, and punishment is not always a trial. In the Old Testament the greatest illustration of this is Job. David has trouble with it in Psalm 73:1–4, 13, 14. Observe again, that the Lord God -— contrary to about 98 percent of contemporary Fundamental and Conservative theology -— hates people. (“Him that loveth violence his soul hateth.”)}
[5] \textcolor[rgb]{0.00,0.00,1.00}{The LORD trieth the righteous: but the wicked and him that loveth violence his soul hateth.}\index[AWIP]{The!Psalms!Psa 011:005}\index[AWIP]{LORD!Psalms!Psa 011:005}\index[AWIP]{trieth!Psalms!Psa 011:005}\index[AWIP]{the!Psalms!Psa 011:005}\index[AWIP]{righteous!Psalms!Psa 011:005}\index[AWIP]{but!Psalms!Psa 011:005}\index[AWIP]{the!Psalms!Psa 011:005 (2)}\index[AWIP]{wicked!Psalms!Psa 011:005}\index[AWIP]{and!Psalms!Psa 011:005}\index[AWIP]{him!Psalms!Psa 011:005}\index[AWIP]{that!Psalms!Psa 011:005}\index[AWIP]{loveth!Psalms!Psa 011:005}\index[AWIP]{violence!Psalms!Psa 011:005}\index[AWIP]{his!Psalms!Psa 011:005}\index[AWIP]{soul!Psalms!Psa 011:005}\index[AWIP]{hateth!Psalms!Psa 011:005}\index[NWIV]{16!Psalms!Psa 011:005}\index[PNIP]{LORD!Psalms!Psa 011:005}
[6] \textcolor[rgb]{0.00,0.00,1.00}{Upon the wicked he shall rain snares, fire and brimstone, and an horrible tempest: \emph{this} \emph{shall} \emph{be} the portion of their cup.}\index[AWIP]{Upon!Psalms!Psa 011:006}\index[AWIP]{the!Psalms!Psa 011:006}\index[AWIP]{wicked!Psalms!Psa 011:006}\index[AWIP]{he!Psalms!Psa 011:006}\index[AWIP]{shall!Psalms!Psa 011:006}\index[AWIP]{rain!Psalms!Psa 011:006}\index[AWIP]{snares!Psalms!Psa 011:006}\index[AWIP]{fire!Psalms!Psa 011:006}\index[AWIP]{and!Psalms!Psa 011:006}\index[AWIP]{brimstone!Psalms!Psa 011:006}\index[AWIP]{and!Psalms!Psa 011:006 (2)}\index[AWIP]{an!Psalms!Psa 011:006}\index[AWIP]{horrible!Psalms!Psa 011:006}\index[AWIP]{tempest!Psalms!Psa 011:006}\index[AWIP]{\emph{this}!Psalms!Psa 011:006}\index[AWIP]{\emph{shall}!Psalms!Psa 011:006}\index[AWIP]{\emph{be}!Psalms!Psa 011:006}\index[AWIP]{the!Psalms!Psa 011:006 (2)}\index[AWIP]{portion!Psalms!Psa 011:006}\index[AWIP]{of!Psalms!Psa 011:006}\index[AWIP]{their!Psalms!Psa 011:006}\index[AWIP]{cup!Psalms!Psa 011:006}\index[NWIV]{22!Psalms!Psa 011:006}
[7] \textcolor[rgb]{0.00,0.00,1.00}{For the righteous LORD loveth righteousness; his countenance doth behold the upright.}\index[AWIP]{For!Psalms!Psa 011:007}\index[AWIP]{the!Psalms!Psa 011:007}\index[AWIP]{righteous!Psalms!Psa 011:007}\index[AWIP]{LORD!Psalms!Psa 011:007}\index[AWIP]{loveth!Psalms!Psa 011:007}\index[AWIP]{righteousness!Psalms!Psa 011:007}\index[AWIP]{his!Psalms!Psa 011:007}\index[AWIP]{countenance!Psalms!Psa 011:007}\index[AWIP]{doth!Psalms!Psa 011:007}\index[AWIP]{behold!Psalms!Psa 011:007}\index[AWIP]{the!Psalms!Psa 011:007 (2)}\index[AWIP]{upright!Psalms!Psa 011:007}\index[NWIV]{12!Psalms!Psa 011:007}\index[PNIP]{LORD!Psalms!Psa 011:007}

%EXAMPLE MAGINNOTE \marginnote{\textcolor[rgb]{0.13, 0.55, 0.13}{\tiny The Early Life of David, in 3 Parts: \begin{compactenum}[I.][7]
%    \item The Praise of his People (Psalm 22:3, specifically Israel)
%        \item The Path of Righteousness
%        \item The Problems we Experience
%        \item His Promises
%        \item Our Prayers to Him
%        \item The Principles of His Word
%        \item The Priority for Souls
%\end{compactenum} } } [0.005cm]



