\section{Psalm 20 Comments}

\subsection{Numeric Nuggets}
\textbf{13:} Verse 8 has 13 words. Verse 1 has 13 unique words.


\subsection{Introduction}


From Ruckman, we get the synopsis of the chapter:
\begin{quote}
The entire Psalm is a Second Advent Psalm. “The day of trouble” (vs. 1) is the“time of trouble” (Dan. 12:1; Matt. 24:21) for Israel. And this time the “God of Jacob” is in business. Help must come from the Jewish sanctuary on Mount Zion (vs. 2), where “burnt sacrifices” (vs. 3) are being offered in the Tribulation (see Rev. 11:1–3). Again the word “Selah” shows up to tell us “where we are at.” “Our God,” and “our banners” show the Jewish context (vs. 5). “Thy petitions” can be aimed at a Gentile who is helping the Jew at this time; note the comments of the Holy Spirit on this matter in Deuteronomy 32:43 and Matthew 25:34–42. “His anointed” (vs. 6) has three applications (exactly  as many of the passages in Ps. 19). It is a reference to David personally and historically; it is a reference to the nation of Israel being delivered in the Tribulation. “His right hand” is a reference to the Lord Jesus (see Ps. 17:7, 139:10). The Antichrist is gathering his troops (vs. 7) for the last attack against Israel (see Zech. 14:1–6; Joel 3:11), but “they are brought down and fallen” (vs. 8) at Armageddon (Rev. 19:20), while it is Israel that rises and stands upright (Isa. 2, 9, 65–66). At that time, Deuteronomy 32:43 will be fulfilled to the jot and tittle. “Let the king hear us when we call” (vs. 9). It is the King of kings—a Jewish King— for “salvation is of the JEWS” (John 4:22). The “us” in the passage is the “us” of Psalm 122 and 124. Nothing is difficult anywhere in the Psalm. The Wycliffe commentator misses the import of every single verse in it, and so do the “New” Bible commentators (Yates and Motyer). \cite{Ruckman1992PsalmsV1}\\
\\
From a devotional standpoint: “May the Lord hear me and defend me’’ (vs. 1). “May the Lord strengthen me and send help”  (vs. 2). “May the Lord remember me and accept me” (vs. 3). And “May the Lord answer me and fulfil my request” (vs. 4. See 21:2). The “name” (vs. 7) has salvation in it (“Jesus”); what need then of “chariots and horses?” Some may substitute “fuel and energy” for “chariots and horses,” while others substitute “tanks and planes”; it comes out the same way. “Let the king hear us when we call.” We don’t want anyone else to answer the phone; we don’t want any “disconnections” while conversing; we don’t want to get a busy signal for twenty minutes at a time, and we do not want any “operator” interrupting our conversation. “Let the king hear us when we call.”
\end{quote}
