\chapter{Psalm 16}
\marginpar{\scriptsize \centering \fcolorbox{black}{yellow}{\textbf{CONTRASTS}}\\ (Psalm 16:1--11) \begin{compactenum}[I.][8]
	\item Thirteen \textbf{Possession} \index[scripture]{Psalms!Psa 016:01}(Psalm 16:1)
	\begin{compactenum}[A.]
	    \item My \textbf{trust} \index[scripture]{Psalms!Psa 016:01}(Psalm 16:1)
	    \item My \textbf{soul} \index[scripture]{Psalms!Psa 016:02}(Psalm 16:2)
	    \item My \textbf{Lord} \index[scripture]{Psalms!Psa 016:02}(Psalm 16:2)
	    \item My \textbf{goodness} \index[scripture]{Psalms!Psa 016:02}(Psalm 16:2)
	    \item My \textbf{delight} \index[scripture]{Psalms!Psa 016:03}(Psalm 16:3)
	    \item My \textbf{lips} \index[scripture]{Psalms!Psa 016:04}(Psalm 16:4)
	    \item My \textbf{cup} \index[scripture]{Psalms!Psa 016:05}(Psalm 16:5)
	    \item My \textbf{lot} \index[scripture]{Psalms!Psa 016:05}(Psalm 16:5)
	    \item My \textbf{right hand} \index[scripture]{Psalms!Psa 016:08}(Psalm 16:8)
	    \item My \textbf{heart} \index[scripture]{Psalms!Psa 016:09}(Psalm 16:9)
	    \item My \textbf{glory} \index[scripture]{Psalms!Psa 016:09}(Psalm 16:9)
	    \item My \textbf{flesh} \index[scripture]{Psalms!Psa 016:09}(Psalm 16:9)
	    \item My \textbf{soul} \index[scripture]{Psalms!Psa 016:10}(Psalm 16:10)
	\end{compactenum}
	\item One \textbf{Provision} (thine Holy One) \index[scripture]{Psalms!Psa 016:10}(Psalm 16:10)
	\item One \textbf{Promise} (mine inheritance) \index[scripture]{Psalms!Psa 016:05}(Psalm 16:5)
\end{compactenum}}



\footnote{\textcolor[rgb]{0.00,0.25,0.00}{\hyperlink{TOC}{Return to end of Table of Contents.}}}\textcolor[cmyk]{0.99998,1,0,0}{Michtam of David.}\\
\\
\textcolor[cmyk]{0.99998,1,0,0}{Preserve me, O God: for in thee do I put my trust.}\footnote{\textbf{Psalm 17:5,8} - Hold up my goings in thy paths, that my footsteps slip not. [8] Keep me as the apple of the eye, hide me under the shadow of thy wings,}\footnote{\textbf{Psalm 31:23} - O love the LORD, all ye his saints: for the LORD preserveth the faithful, and plentifully rewardeth the proud doer.}\footnote{\textbf{Psalm 37:28} - For the LORD loveth judgment, and forsaketh not his saints; they are preserved for ever: but the seed of the wicked shall be cut off.}\footnote{\textbf{Psalm 56:1} - Be merciful unto me, O God: for man would swallow me up; he fighting daily oppresseth me.}\footnote{\textbf{Psalm 60:1} - O God, thou hast cast us off, thou hast scattered us, thou hast been displeased; O turn thyself to us again.}\footnote{\textbf{Psalm 97:10} - Ye that love the LORD, hate evil: he preserveth the souls of his saints; he delivereth them out of the hand of the wicked.}\footnote{\textbf{Psalm 116:6} - The LORD preserveth the simple: I was brought low, and he helped me.}\footnote{\textbf{Proverb 2:8} - He keepeth the paths of judgment, and preserveth the way of his saints.}
[2] \textcolor[cmyk]{0.99998,1,0,0}{\emph{O} \emph{my} \emph{soul}, thou hast said unto the LORD, Thou \emph{art} my Lord: my goodness \emph{extendeth} not to thee;}
[3] \textcolor[cmyk]{0.99998,1,0,0}{\emph{But} to the saints that \emph{are} in the earth, and \emph{to} the excellent, in whom \emph{is} all my delight.}\footnote{``My goodness extendeth not to thee'' if spoken by Christ (and verses 8–-11 certainly are), can only mean that the Lord Jesus’ righteousness does not have to be given to God the Father: it must be given to the ``saints,'' for note what follows: ``But to the saints that are in the earth.'' This is a clear foreshadowing of future ``imputed righteousness'' (Romans 10:1–8).\cite{Ruckman1992Psalms}}
[4] \textcolor[cmyk]{0.99998,1,0,0}{Their sorrows shall be multiplied \emph{that} hasten \emph{after} another \emph{god}: their drink offerings of blood will I not offer, nor take up their names into my lips.}\footnote{“Their drink offerings of blood will I not offer.” There went Ted Kennedy, Rock Hudson, Frank Sinatra, Bing Crosby, Tip O’Neill, Pope John Paul II, Bernadette Devlin, Adolph Hitler, Pope John XXIII, Mussolini, Fidel Castro, Pope Pius XII, Grace Kelly, Jack Kennedy, Pope John VI, and your local “father” out the window and into Gehenna. Christ would not offer literal BLOOD as “DRINK offerings” if He were on this earth today. If your priest does, he is in an outfit that Jesus Christ would not mention verbally: “nor take up their names into my lips.” A drink that is literal BLOOD is forbidden in three dispensations: before the Law (Gen. 9:4), under the Law (Lev. 17:10), and after the Law (Acts 15:20). A “drink offering” is offered as a sacrifice in Leviticus, so the most damnable, unholy, ungodly thing any professing Christian could do would be to call the Lord’s supper a “SACRIFICE” (which it is not) and then offer a libation of literal BLOOD to drink. You couldn’t blaspheme the Holy Spirit any more effectively than that if you were the most depraved atheist who ever defiled the word or words of God. Isaiah 65:11 and Isaiah 6:13 told you that literal flesh and blood would be eaten and drunk in the Tribulation, and that is why the bloody drink is mentioned here: it just followed Psalm 14:4. The Psalms are hard on the Papists and Mariolators; we are told in Psalm 69:8 that Mary had other physical children, which she begat herself after Jesus was born. Psalm 69:8 and Psalm 16:4 were given by God the Holy Spirit to warn you that every Catholic priest on this earth is a pathological, religious LIAR: every one one of them, without one single exception in fifteen hundred years.\cite{Ruckman1992Psalms}} \footnote{The verse, then, connects drink offerings, with blood (and literal blood), and ``another god.'' Does this make one think of any particular religious group? }
[5] \textcolor[cmyk]{0.99998,1,0,0}{The LORD \emph{is} the portion of mine inheritance and of my cup: thou maintainest my lot.}
[6] \textcolor[cmyk]{0.99998,1,0,0}{The lines are fallen unto me in pleasant \emph{places}; yea, I have a goodly heritage.}\footnote{The idea is a piece of land has been measured out with a “line” (see Ezek. 40–41, and 42:16; Amos 7:17; and Zechariah 2:1). Christ’s ``goodly heritage'' is \cite{Ruckman1992Psalms}:
\begin{compactenum}
\item A body of believers who will be exactly like Him someday.
\item The angels in subjection to Him as King of Kings.
\item The earth and the fullness thereof (Ps. 72).
\item The literal land of Palestine where He was crucified.
\item David’s “throne of glory” at Jerusalem.
\item The glory God gave Him before the world began (John 17:5).
\end{compactenum} }
[7] \textcolor[cmyk]{0.99998,1,0,0}{I will bless the LORD, who hath given me counsel: my reins also instruct me in the night seasons.}
[8] \textcolor[cmyk]{0.99998,1,0,0}{I have set the LORD always before me: because \emph{he} \emph{is} at my right hand, I shall not be moved.}
[9] \textcolor[cmyk]{0.99998,1,0,0}{Therefore my heart is glad, and my glory rejoiceth: my flesh also shall rest in hope.}
[10] \textcolor[cmyk]{0.99998,1,0,0}{For thou wilt not leave my soul in hell; neither wilt thou suffer thine Holy One to see corruption.}
[11] \textcolor[cmyk]{0.99998,1,0,0}{Thou wilt shew me the path of life: in thy presence \emph{is} fulness of joy; at thy right hand \emph{there} \emph{are} pleasures for evermore.}\footnote{One may ask himself, “Who is holding the REINS in my life?” (vs. 7). The “night seasons” are the most dangerous seasons. The Lord should be set “before” us (vs. 8) first of all for leadership; secondly, so that we can observe Him “looking unto,” (Heb. 12:2); and thirdly, so we can take a second choice after He has taken the first choice. The “path of life” (vs. 11) goes through death; that is instructive. “At thy right hand there are pleasures for evermore.” Since the pleasures of sin are only “for a season” (Heb. 11:25), and most of you are “lovers of pleasures” (2 Tim. 3:4), why not get ETERNAL pleasures? Why not enjoy yourself forever instead of just for seventy to one hundred years? Psalm 16:11 goes with Psalm 21:6. \cite{Ruckman1992Psalms} } \footnote{The great thing about ``pleasures for evermore'' escapes the eye of all of the theologians and exegetes. They lack a ``childlikeness'' (not ``childishness''), which often hinders their comprehension of great Biblical truths. The truth is, in New Jerusalem, any saved sinner can think anything he wants to think, say anything he wants to say and DO anything he wants to do—ANYTHING, anything that comes into his imagination—without ever having to check it once to see if it is “right” or “wrong.” That is the “glorious liberty” of the children of God (Rom. 8:21). All of the scholars and theologians miss it. If they were children they would grasp THAT truth immediately; it would be the first truth about heaven they would grasp: any child wants to do what he wants to do when he wants to do it. \cite{Ruckman1992Psalms} }





