\subsection{Outlines from Campbell}

\subsubsection{Psalm 22, 23, 24}
%\textbf{Introduction:} Psalm 2:\footnote{8 June 2016, Keith Anthony}
\index[speaker]{Joe Campbell!Psalm 022 (Psalm 22, 23, 24)}
\index[series]{Psalms (Joe Campbell)!Psalm 022 (Psalm 22, 23, 24)}
\index[date]{2013/03/12!Psalm 022 (Psalm 22, 23, 24) (Joe Campbell)}

\index{FACEBOOK}{ALLITERATED SERMON OUTLINES!Joe Campbell (Psalm 22, 23, 24) - Psalm 22 (2013/03/12)}

​I.The Cross: Psalm 22 --The suffering Saviour
II.The Crook: Psalm 23 -- The living Shepherd
III.The Crown: Psalm 24 -- The exalted Sovereign

Remember
Psalm 22:25-29
Intro.
This is Memorial Day week-end
It’s sad that many do not know nor recognize what this day
was for.
It was to be a day for mourning and remembering those
who had died in battle.
Memorial Day was started after the civil war the remember
those who had did in battle.
Different states held it on different days.
On the first national memorial day was on May 30, 1868 it was proclaimed that it was “for the purpose of strewing with flowers or otherwise decorating the graves of comrades who died in defense of their country during the late rebellion, and whose bodies now lie in almost every city, village, and hamlet churchyard in the land.
It was not made a National Holiday until 1971
We used to call it “Decoration Day”
Memorial Day should be about remembering
I. Remembering the people – Those who gave their lives.
Vs. 29 “they that go down to the dust”
Here is a review of those who died in American wars.
Revolutionary War: 4425 Deaths
War of 1812: 2260 Deaths
Second Seminole War: The American army suffered 1466 deaths, 328 were killed in action, 1138 died by disease. There was also 69 deaths in the navy.
No one seems to know how many Seminoles died in the war, but over 3000 were taken captive and moved to the west.
Civil War: Approximately 620,000 Americans
died. The Union lost almost 365,000 troops and the
Confederacy about 260,000. More than half of these deaths were caused by disease.
World War I: 116,516 Americans died, more than half from disease.
World War II: 405,399 Americans died.
Korean War: 36,574 Americans died.
Vietnam Conflict: 58,220 Americans died. More than
47,000 Americans were killed in action and nearly 11,000 died of other causes.
Memorial Day was set aside to remember all those who had died in battle.
Proclamation by Governor Ron DiSantis (2020)
*Find the proclamation for your respective State.
It was NOT intended that Memorial Day should be
observed by parties, and bar-b-ques, etc.
It was to be a day of mourning and remembering
II. Remembering the price – Death
A. Memorial Day in not to be compared to Veterans day.
Veteran’s Day is for the living.
Memorial Day is for the fallen.
B. The price was high – these soldiers gave their life for
their country.
C. They highest price that could be paid. A price that cannot be repaid.
1. We cannot give them their life back.
2. We can only give them honor, and respect.
3. That is why the flag should fly at half staff till noon.
III. Remember the principle – Freedom
A. In every war the principle was the same – Freedom.
B. When our solders went over seas, they were not fighting
for expansion.
C. They were not fighting to take over other nations.
D. Our soldiers died defending our freedom.
1. We should never take for granted.
2. We should never forget the price that was paid.
3. Our reaction - Psalm 22:26
a. We should be humbled. (meek)
b. We should praise the Lord.
IV. Remember the prize -- Victory
A. The sacrifice of our fallen warriors has led us to victory.
B. They didn’t retreat
C. They didn’t give up.
D. They didn’t quit.
E. They fought till the victory was won.
Conclusion:
Take time to remember those who died for you.
Take time to reflect on the freedoms you have.
Take time to return to the Lord. Vs. 27
Take time to raise your voice in praise to God. Vs. 27