\chapter{Psalm 18}
\footnote{\textcolor[rgb]{0.00,0.25,0.00}{\hyperlink{TOC}{Return to end of Table of Contents.}}}\textcolor[rgb]{0.00,0.00,1.00}{To the chief Musician, \emph{A Psalm} of David, the servant of the LORD, who spake unto the LORD the words of this song in the day \emph{that} the LORD delivered him from the hand of all his enemies, and from the hand of Saul: And he said,}\footnote{This is one of the greatest Messianic Psalms in the entire collection. The Psalm is almost (but not quite) a reproduction of 2 Samuel 22, which you should see. The minute variations show us three things that the manuscriptolators don’t want you to notice \cite{Ruckman1992Psalms}:\begin{compactenum}
\item Two inspired accounts of the same events do not have to match word for word.
\item Because two words in two identical accounts do not match does not mean there is a contradiction in EITHER of them.
\item A copyist can ``miscopy'' a word and it still be INSPIRED. It is not just that the English text is not word-for-word; it is not word-for-word in ANY Masoretic text from ANY set of Hebrew manuscripts.\end{compactenum}
The only way you can avoid this conclusion (no. 3) is to say that ALL of the Hebrew manuscripts for 2 Samuel chapter 22 are in error or ALL of the Hebrew manuscripts for Psalm 18 are in error. It won’t do you any good then because Matthew, Mark, Luke, and John do not give matching accounts of a dozen events.}\\
\\

[19] \footnote{Commentators apply the whole passage to David in the past. If he is going to “stick by his guns” then he has proved that under the law there is an element of personal righteousness and WORKS that have to do with a man’s salvation: “My righteousness...the cleanness of my hands...I have kept the ways of the Lord...I did not put away his statutes...I was also upright before him...the Lord recompensed me according to my righteousness.” You couldn’t find Ephesians chapter 2 or John 3:16 in ONE line of it. Such a statement showing up after the Council of Jerusalem (Acts 15) would have branded the speaker as a LOST PHARISEE.
Once the passage is applied to Jesus Christ then He has to keep Himself from His own “iniquity” (vs. 23). Impossible. So the whole chapter is impossible to a modern critic of the AV. The dual application breaks the seat belt and hurtles him through the windshield. There is nothing like unbelief to wreck a scholar. The truth is, David, as a type of Christ, is speaking poetically of his own victories while at the same time describing the literal Second Advent of Jesus Christ. (Look at the same thing in Psalms 76, and 91). Occasionally David’s nature will “peek through,” as in verse 23 and verse 50, but as in Psalm 22 and Psalm 69, the Lord Jesus Christ Himself is the subject. Note, in Psalm 69, that if you applied the whole Psalm to Jesus Christ, He then would have “sins” and “foolishness” (Ps. 69:5): moreover, if you applied all of Psalm 22 to David, he would be lying like a dog because no one ever “pierced” his hands or his feet neither “cast lots” for his vesture. \cite{Ruckman1992Psalms}}