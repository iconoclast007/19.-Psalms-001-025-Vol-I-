\section{Psalm 10 Comments}
A recent conversation revealed to me that some read Psalms and individual  psalms merely as poetry. So, words can mean almost anything. However, scripture with scripture provides insight into Bible interpretation, since the Bible is not a normal book. Psalm 10 presents us with a choice: is  or are ``the wicked'' a specific individual or a group of people whose ways as characterized as wicked? Or is the first part of verse 2 speaking of an individual, and the second part a group of people.  For example, 2 Thessalonians 2:8 speaks of ``that Wicked'' while 2 Thesalonians 3:2 speaks ``wicked men.''  There are 344 occurrences of the word ``wicked'' in scripture, some referring to a specific wicked person.  Here in Psalm 10, after verse 1, there are 32 uses of the words \emph{his}, ``his'', ``his'', ``He'', ``he'',  and ``His'', there to remove confusion on the issue.\\
\\
\noindent Another choice would be the second part of verse 2 referring to 2 or 3 members of the Satanic Trinity (The Beast, the False Prophet, and the Antichrist). \\
\\
\noindent Several commentaries are of note. Ruckman wastes no time and dives straight into the essence of the psalm, the tribulation, the antichrist, and his ruin. Phillips is among the few that also recognize the focus of the psalm, even to the point of recognizing the individuals in the Satanic trinity.\cite{Phillips2001ExploringPsalms1}   Hodge, in his commentary (available on Google Books), provides references for this individual in Daniel 11:36,37, the Antichrist.\footnote{\textbf{Daniel 11:36-39} - And the king shall do according to his will; and he shall exalt himself, and magnify himself above every god, and shall speak marvellous things against the God of gods, and shall prosper till the indignation be accomplished: for that that is determined shall be done. [37] Neither shall he regard the God of his fathers, nor the desire of women, nor regard any god: for he shall magnify himself above all. [38] But in his estate shall he honour the God of forces: and a god whom his fathers knew not shall he honour with gold, and silver, and with precious stones, and pleasant things. [39] Thus shall he do in the most strong holds with a strange god, whom he shall acknowledge and increase with glory: and he shall cause them to rule over many, and shall divide the land for gain}.\cite{deBurgh1858PsalmCommentary}\\
\\
\noindent By avoiding, denying,  removing, or spiritualizing the reference to ``Wicked'' as the antichrist,  commentators feel free to make the ``wicked'' into a group of oppressors. See for example, the work in the Understanding the Bible Commentary series by Broyles. Broyles also points out the connections between Psalm 9 and 10, but jumps ship by claiming the two psalms are dealing mainly with social oppression. \cite{broyles2012psalms}. Walvoord at least stays safe and vague by discussing ``the oppressor'' and ``the wicked person'', but ends with the even more vague, ``the afflicted'' and ``the tyranny of the wicked.'' \cite{walvoord1985bibleknowledgeOT}\\
\\
Devotionally, Tim Keller in \emph{The Songs of Jesus} gives us some comfort. We can choose the noble and righteous path, despite outward appearances, or follow the self-indulgent and destructive ways of wickedness. At the end of the day, God hears the cries of the afflicted and oppressed and removes evil. Affliction is temporary, but judgment is permanent. \cite{keller2015songs}\\
\\
Commentators typically connect Psalm 9 and 10, as they are in the LXX, or even argue that the two together were part of a now lost bigger psalm. I argue that the separattion of the two psalms is one of the ``Dispensational Commas'' which separate the Old Testament from the time when Israel is tested, repents, and is reborn. This comma includes the Church Age.\cite{wilcock2001MessageOfPsalms1to72}

\subsection{Numeric Nuggets}
The word ``in'' in used 13 times in the psalm, the 13$^{th}$ reference in verse 13 (See Table~\ref{table:WhenAndWhereOfTheWicked}). Verse 12 contains 13 words. The word ``wicked'' shows up 5 times in the chapter, connecting this individual with death.


\begin{table}[]
\begin{center}
\begin{tabular}{|c|c|c|}
\hline 
\#(s) & verse(s) & text \\ \hline 
1 & 1 & ``in times of trouble'' \\ \hline 
2 & 2 & ``in his pride'' \\ \hline 
3 & 2 & ``in the devices'' \\ \hline 
4 & 4 & ``(not) in all his thoughts'' \\ \hline 
5, 12, 13 & 6, 11, 13 & ``in  his heart'' \\ \hline 
6 & 6 & ``in adversity'' \\ \hline 
7 & 8 & ``in the lurking places'' \\ \hline 
8 & 8 & ``in the secret places'' \\ \hline 
9,10 & 9, 9 & ``in wait'' \\ \hline 
11 & 11 & ``in his den'' \\ \hline 
\end{tabular}

\label{table:WhenAndWhereOfTheWicked}
\caption[The When and Where of the Wicked]{The When and Where of the Wicked}
\end{center}
\end{table}


\subsection{Psalm 10:1}
There is only one other occurrence of the phrase ``times of trouble'', in Psalm 9:9. But Time of trouble occurs eight times ((1) Job 38:23, (2) Psalms 27:5, (3) Psalms 37:39, (4) Psalms 41:1, (5) Proverbs 25:19, (6) Isaiah 33:2, (7) Jeremiah 14:8, and (8) Daniel 12:1). The verses speaks of God's provision for his people during this time. The ``time of Jacob's trouble'' is mentioned in Jeremiah 30:7.\footnote{\textbf{Jeremiah 30:7} - Alas! for that day is great, so that none is like it: it is even the time of Jacob’s trouble; but he shall be saved out of it.}

\subsection{Psalm 10:2}
The verse is corrupted in modern transaltions, as seen in Table~\ref{table:Corruption Psalm 10:2}.

\begin{center}

\begin{table}[ht]
\centering
\begin{tabular}{|p{.5in}|p{3.5in}|}
    \hline

\textcolor[rgb]{0.00,0.00,1.00}{AV} & \textcolor[rgb]{0.00,0.00,1.00}{The \fcolorbox{blue}{lime}{wicked} in \emph{his} pride doth persecute the poor: let them be taken in the devices that they have imagined.} \\ \hline
%
ASV &  In the pride of the wicked the poor is hotly pursued; Let them be taken in the devices that they have conceived. \\ \hline
%
CEB &  Meanwhile, the wicked are proudly   in hot pursuit of those who suffer. Let them get caught  in the very same schemes they’ve thought up!\\ \hline
%
ESV & In arrogance the wicked hotly pursue the poor; let them be caught in the schemes that they have devised. \\ \hline
%
NASV &  In pride the wicked hotly pursue the afflicted; Let them be caught in the plots which they have devised.\\ \hline
%
MEV & In arrogance the wicked persecutes the poor;   let them be caught in the devices they have planned.\\ \hline
%
NIV &  In his arrogance the wicked man hunts down the weak, who are caught in the schemes he devises. \\ \hline
%
NKJV &  The wicked in his pride persecutes the poor; Let them be caught in the plots which they have devised.\\ \hline
%
RSV &  In arrogance the wicked hotly pursue the poor;   let them be caught in the schemes which they have devised.\\ \hline \hline

\multicolumn{2}{|p{4.3in}|}{{\textcolor{jungle}{Modern translations obscure the reference to the Antichrist, identified elsewhere in scripture as the ``Wicked''.  Renderings in the modern versions seem to be speaking of a group, referred to as ``the wicked.'' Even the NIV can be read as a man in general or man in general as opposed to a specific person. The AV reading further refers to a singular man, but then a plural (``them'') is used. The second part of verse 2 could be about the followers of the man named in the first part. NIV completely botches the second part of the verse. Consider, further, the differences between devices, schemes, and plots.\cite{Ruckman1992PsalmsV1}.}}} \\ \hline

\end{tabular}
\caption[Corruption Alert: Psalm 10:2]{Corruption Alert: Psalm 10:2} \label{table:Corruption Psalm 10:2}

\end{table}

\end{center}


\subsection{Psalm 10:5}
The verse is also corrupted in modern transaltions, as seen in Table~\ref{table:Corruption Psalm 10:5}.

\begin{center}

\begin{table}[ht]
\centering
\begin{tabular}{|p{.5in}|p{3.5in}|}
    \hline

\textcolor[rgb]{0.00,0.00,1.00}{AV} & \textcolor[rgb]{0.00,0.00,1.00}{His ways are always grievous; thy judgments are far above out of his sight: as for all his enemies, he puffeth at them.} \\ \hline
%
ASV &  His ways are firm at all times; Thy judgments are far above out of his sight: As for all his adversaries, he puffeth at them. \\ \hline
%
CEB &  Their ways are always twisted.  Your rules are too lofty for them.   They snort at all their foes. \\ \hline
%
ESV & His ways prosper at all times;  your judgments are on high, out of his sight;   as for all his foes, he puffs at them. \\ \hline
%
NASV &  His ways prosper at all times; Your judgments are on high, out of his sight;  As for all his adversaries, he snorts at them.\\ \hline
%
MEV & His ways are always prosperous; Your judgments are high and distant from him;   as for all his enemies, they scoff at him. \\ \hline
%
NIV & His ways are always prosperous;   your laws are rejected by him;    he sneers at all his enemies. \\ \hline
%
NKJV &  His ways are always prospering; Your judgments are far above, out of his sight; As for all his enemies, he sneers at them.\\ \hline
%
RSV & His ways prosper at all times;   thy judgments are on high, out of his sight;  as for all his foes, he puffs at them.\\ \hline

\multicolumn{2}{p{4.3in}}{{Modern versions change the grievous works of ``the wicked'' to ``prosperous'', removing all the cross references in scripture (variants of grievous occur 47 times in the AV).  From the point of view of God's people, all the works of ``the Wicked'' are grievous, and not prosperous.}} \\ %\hline
\end{tabular}

\caption[Corruption Alert: Psalm 10:5]{Corruption Alert: Psalm 10:5} \label{table:Corruption Psalm 10:5}

\end{table}

\end{center}






\subsection{Psalm 10:9}
This lion is the lion of 1 Peter 5:8.\footnote{\textbf{1 Peter 5:8} - Be sober, be vigilant; because your adversary the devil, as a roaring lion, walketh about, seeking whom he may devour:} Here and in verse 10, the ``poor'' show up three times. At this point in time, then, it appears that the ``righteous'' and ``godly'' will be poor. We see this in numerous verses, such as Psalm 12:5\footnote{\textbf{Psalm 12:5} - For the oppression of the poor, for the sighing of the needy, now will I arise, saith the LORD; I will set him in safety from him that puffeth at him.}, Psalm 14:6\footnote{\textbf{Psalm 14:6} - Ye have shamed the counsel of the poor, because the LORD is his refuge.}, Proverb 14:21\footnote{\textbf{Proverb 14:21} - He that despiseth his neighbour sinneth: but he that hath mercy on the poor, happy is he.}, Proverb 19:1\footnote{\textbf{Proverb 19:1} - Better is the poor that walketh in his integrity, than he that is perverse in his lips, and is a fool.}, Ecclesiastes 5:8\footnote{\textbf{Ecclesiastes 5:8} - If thou seest the oppression of the poor, and violent perverting of judgment and justice in a province, marvel not at the matter: for he that is higher than the highest regardeth; and there be higher than they.}, Isaiah 32:7\footnote{\textbf{Isaiah 32:7} - The instruments also of the churl are evil: he deviseth wicked devices to destroy the poor with lying words, even when the needy speaketh right.}, Isaiah 66:2\footnote{\textbf{Isaiah 66:2} - For all those things hath mine hand made, and all those things have been, saith the LORD: but to this man will I look, even to him that is poor and of a contrite spirit, and trembleth at my word.}, Amos 8:4\footnote{\textbf{Amos 8:4} - Hear this, O ye that swallow up the needy, even to make the poor of the land to fail,}, and Zechariah 11:7, 11.\footnote{\textbf{Zechariah 11:7, 11} - And I will feed the flock of slaughter, even you, O poor of the flock. And I took unto me two staves; the one I called Beauty, and the other I called Bands; and I fed the flock. [11] And it was broken in that day: and so the poor of the flock that waited upon me knew that it was the word of the LORD.} This is certainly not the case in the present age, and certainly not true in the United States (although there is ``smoke on the horizon''). There are plenty who preach that ``gain is godliness'' (1 Timothy 6:5), and those who carry the ``health and wealth'' gospel. We see the `poor'' and ``fatherless'' again in verse 14.

\subsection{Psalm 10:15}
The reference is to the Antichrist in Zechariah 11;17, who is wounded and loses the use of his right eye. See also Jeremiah 48:25. The reference to the arm is alluded to also in Ezekiel 30:2 and 24, speaking of Pharaoh who typifies the Antichrist.\footnote{\textbf{Zechariah 11:17} -  Woe to the idol shepherd that leaveth the flock! the sword shall be upon his arm, and upon his right eye: his arm shall be clean dried up, and his right eye shall be utterly darkened.}\footnote{\textbf{Jeremiah 48:25} - The horn of Moab is cut off, and his arm is broken, saith the LORD.}\footnote{\textbf{Ezekiel 30:2, 24} Son of man, prophesy and say, Thus saith the Lord GOD; Howl ye, Woe worth the day! [24] And I will strengthen the arms of the king of Babylon, and put my sword in his hand: but I will break Pharaoh’s arms, and he shall groan before him with the groanings of a deadly wounded man.} The ``Wicked'' is spoken of in 2 Thessalonians 2:8.\footnote{\textbf{2 Thessalonians 2:8} - And then shall that Wicked be revealed, whom the Lord shall consume with the spirit of his mouth, and shall destroy with the brightness of his coming.} For further details, see the references in 1 John 2:13, 2:14, 3:12, and 5:18. One should study the 344 references to ``wicked'' in scripture. Pictures can be seen of this man as ``Popeye'' in media. In history, we see him as Moshe Dayan, as Napolean, as Kaiser Wilhelm, as Bob Dole, all with disabled right arms.\cite{Ruckman1992Psalms}


