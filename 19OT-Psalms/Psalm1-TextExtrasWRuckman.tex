\chapter{Psalm 1}

\footnote{\textcolor[rgb]{0.00,0.25,0.00}{\hyperlink{TOC}{Return to end of Table of Contents.}}}\textcolor[rgb]{0.00,0.00,1.00}{Blessed is the man that walketh not in the counsel of the ungodly, nor standeth in the way of sinners, nor sitteth in the seat of the scornful.}\index[AWIP]{Blessed!Psalms!Psa 001:01}\index[AWIP]{is!Psalms!Psa 001:01}\index[AWIP]{the!Psalms!Psa 001:01}\index[AWIP]{man!Psalms!Psa 001:01}\index[AWIP]{that!Psalms!Psa 001:01}\index[AWIP]{walketh!Psalms!Psa 001:01}\index[AWIP]{not!Psalms!Psa 001:01}\index[AWIP]{in!Psalms!Psa 001:01}\index[AWIP]{the!Psalms!Psa 001:01 (2)}\index[AWIP]{counsel!Psalms!Psa 001:01}\index[AWIP]{of!Psalms!Psa 001:01}\index[AWIP]{the!Psalms!Psa 001:01 (3)}\index[AWIP]{ungodly!Psalms!Psa 001:01}\index[AWIP]{nor!Psalms!Psa 001:01}\index[AWIP]{standeth!Psalms!Psa 001:01}\index[AWIP]{in!Psalms!Psa 001:01 (2)}\index[AWIP]{the!Psalms!Psa 001:01 (4)}\index[AWIP]{way!Psalms!Psa 001:01}\index[AWIP]{of!Psalms!Psa 001:01 (2)}\index[AWIP]{sinners!Psalms!Psa 001:01}\index[AWIP]{nor!Psalms!Psa 001:01 (2)}\index[AWIP]{sitteth!Psalms!Psa 001:01}\index[AWIP]{in!Psalms!Psa 001:01 (3)}\index[AWIP]{the!Psalms!Psa 001:01 (5)}\index[AWIP]{seat!Psalms!Psa 001:01}\index[AWIP]{of!Psalms!Psa 001:01 (3)}\index[AWIP]{the!Psalms!Psa 001:01 (6)}\index[AWIP]{scornful!Psalms!Psa 001:01}\index[NWIV]{28!Psalms!Psa 001:01}\footnote{From Genesis 30:13 we learn that the word ``blessed'' means ``happy''. Someone has correctly observed that \cite{Ruckman1992Psalms}:\begin{compactitem}
\item Job is the Unhappy Man
\item Psalms is the Happy Man
\item Proverbs is the Wise Man
\item Ecclesiastes is the Worldly Man, 
\item and Song of Solomon is the Heavenly Man.
\end{compactitem}} \footnote{\textbf{Genesis 30:13} -- And Leah said, Happy am I, for the daughters will call me blessed: and she called his name Asher. \cite{Ruckman1992Psalms}} \footnote{There are a number of things that are promised to the Old Testament ``blessed man'' for he is ``planted'' in the right place. \cite{Ruckman1992Psalms}:\begin{compactenum}
\item Good company.
\item Good books (Gen.–Ruth).
\item Good fruit (vs. 3).
\item Good success (vs. 3).
\item Good health.
\item  Good thoughts (vs. 2).
\item One may add; a good position (vs. 3)
\end{compactenum} } \footnote{The words ``blessed'' and ``man'' are used together in 22 verses in scripture, ten of these verses in Psalms.  Letting the Bible interpret itself, then:\begin{compactenum}
    \item (Psalm 1:1) Here
    \item (Psalm 32:2) a man to whom the Lord does not impute sin
    \item (Psalm 34:8) a man that trusts in the Lord
    \item (Psalm 40:4) a man that trusts in the Lord
    \item (Psalm 65:4) a man that God chooses
    \item (Psalm 84:5) a man whose strength is in the Lord
    \item (Psalm 84:12) a man that trusts in the Lord
    \item (Psalm 94:12) a man whom the Lord chastens
    \item (Psalm 112:1) a man who fears the Lord
    \item (Psalm 128:4)  a man who fears the Lord
\end{compactenum} }
[2] \textcolor[rgb]{0.00,0.00,1.00}{But his delight is in the law of the LORD; and in his law doth he meditate day and night.}\index[AWIP]{But!Psalms!Psa 001:02}\index[AWIP]{his!Psalms!Psa 001:02}\index[AWIP]{delight!Psalms!Psa 001:02}\index[AWIP]{is!Psalms!Psa 001:02}\index[AWIP]{in!Psalms!Psa 001:02}\index[AWIP]{the!Psalms!Psa 001:02}\index[AWIP]{law!Psalms!Psa 001:02}\index[AWIP]{of!Psalms!Psa 001:02}\index[AWIP]{the!Psalms!Psa 001:02 (2)}\index[AWIP]{LORD!Psalms!Psa 001:02}\index[AWIP]{and!Psalms!Psa 001:02}\index[AWIP]{in!Psalms!Psa 001:02 (2)}\index[AWIP]{his!Psalms!Psa 001:02 (2)}\index[AWIP]{law!Psalms!Psa 001:02 (2)}\index[AWIP]{doth!Psalms!Psa 001:02}\index[AWIP]{he!Psalms!Psa 001:02}\index[AWIP]{meditate!Psalms!Psa 001:02}\index[AWIP]{day!Psalms!Psa 001:02}\index[AWIP]{and!Psalms!Psa 001:02 (2)}\index[AWIP]{night!Psalms!Psa 001:02}\index[NWIV]{20!Psalms!Psa 001:02}\index[PNIP]{LORD!Psalms!Psa 001:02} \footnote{The antidote to verse 1 is verse 2. If you do verse 2, you will not be found committing the bad things in verse 1. Separation is not just negative (see Romans 1:1). In the Old Testament the Psalms say that the way to happiness is to be found in a triplet: (1) be careful where you WALK, (2) be careful where you STAND, and (3) be careful where you SIT.  For the New Testament believer, his walk today should be ``in him'' (Colossians 2:6); his stand should be ``in the power of his might'' (Ephesians 6:10); and his seat is ``in heavenly places'' (Ephesians 2:6). \cite{Ruckman1992Psalms}} \footnote{To ``delight in the law of the Lord'' is to esteem it more than your ``necessary food'' (according to Job 23:12) and to rejoice over it the same way you would rejoice at finding valuable treasure or spoil. (Psalm 119:162).\cite{Ruckman1992Psalms}} \footnote{\textbf{Job 23:12} -- Neither have I gone back from the commandment of his lips; I have esteemed the words of his mouth more than my necessary \emph{food}.} \footnote{\textbf{Psalm 119:162} -- I rejoice at thy word, as one that findeth great spoil.} \footnote{Contrary to the concept of empty minded meditation, the Bible provides clear things upon which man should meditate, the word being used 14 times in scripture: (1) Genesis 24:63, (2) Joshua 1:8, (3) Psalm 1:2, (4) Psalm 63:6, (5) Psalm 77:12, (6) Psalm 119:15, (7) Psalm 119:23, (8) Psalm 119:48, (9) Psalm 119:78, (10) Psalm 119:148, (11) Psalm 143:5, (12) Isaiah 33:18, (13) Luke 21:14, (14) 1 Timothy 4:15.}
[3] \textcolor[rgb]{0.00,0.00,1.00}{And he shall be like a tree planted by the rivers of water, that bringeth forth his fruit in his season; his leaf also shall not wither; and whatsoever he doeth shall prosper.}\index[AWIP]{And!Psalms!Psa 001:03}\index[AWIP]{he!Psalms!Psa 001:03}\index[AWIP]{shall!Psalms!Psa 001:03}\index[AWIP]{be!Psalms!Psa 001:03}\index[AWIP]{like!Psalms!Psa 001:03}\index[AWIP]{a!Psalms!Psa 001:03}\index[AWIP]{tree!Psalms!Psa 001:03}\index[AWIP]{planted!Psalms!Psa 001:03}\index[AWIP]{by!Psalms!Psa 001:03}\index[AWIP]{the!Psalms!Psa 001:03}\index[AWIP]{rivers!Psalms!Psa 001:03}\index[AWIP]{of!Psalms!Psa 001:03}\index[AWIP]{water!Psalms!Psa 001:03}\index[AWIP]{that!Psalms!Psa 001:03}\index[AWIP]{bringeth!Psalms!Psa 001:03}\index[AWIP]{forth!Psalms!Psa 001:03}\index[AWIP]{his!Psalms!Psa 001:03}\index[AWIP]{fruit!Psalms!Psa 001:03}\index[AWIP]{in!Psalms!Psa 001:03}\index[AWIP]{his!Psalms!Psa 001:03 (2)}\index[AWIP]{season!Psalms!Psa 001:03}\index[AWIP]{his!Psalms!Psa 001:03 (3)}\index[AWIP]{leaf!Psalms!Psa 001:03}\index[AWIP]{also!Psalms!Psa 001:03}\index[AWIP]{shall!Psalms!Psa 001:03 (2)}\index[AWIP]{not!Psalms!Psa 001:03}\index[AWIP]{wither!Psalms!Psa 001:03}\index[AWIP]{and!Psalms!Psa 001:03}\index[AWIP]{whatsoever!Psalms!Psa 001:03}\index[AWIP]{he!Psalms!Psa 001:03 (2)}\index[AWIP]{doeth!Psalms!Psa 001:03}\index[AWIP]{shall!Psalms!Psa 001:03 (3)}\index[AWIP]{prosper!Psalms!Psa 001:03}\index[NWIV]{33!Psalms!Psa 001:03}\footnote{Jeremiah 17:8 is the perfect commentary on Psalm 1:1--3. \cite{Ruckman1992Psalms}} \footnote{\textbf{Jeremiah 17:8} -- For he shall be as a tree planted by the waters, and \emph{that} spreadeth out her roots by the river, and shall not see when heat cometh, but her leaf shall be green; and shall not be careful in the year of drought, neither shall cease from yielding fruit.} \footnote{There are three qualities of the Holy Spirit mentioned in verse 3.  \cite{Ruckman1992Psalms} \begin{compactenum}
\item His similarity to water as a purifying thirst quencher, \item His ability to bear fruit in the ``branch'', 
\item and His ability to prosper what a believer does.
\end{compactenum} Note the Old Testament works setup; personal righteousness versus personal ungodliness. We can make spiritual and devotional application, but the righteous man here is righteous because of his conduct, not because of the imputed righteousness of a risen Saviour.  The ungodly man here is ungodly because of his ``life style,'' and this will be found to be the case throughout the entire book of Job and the entire book of Proverbs.  Despite commentators' efforts to force the Psalms into the Pauline Epistles (which can be done devotionally) they typically have nothing to do with sound doctrine.} \footnote{The phrase ``rivers of water'' is found here in Psalm 1:3, Proverbs 21:1, Isaiah 32:2, and Lamentations 3:48. Interestingly, Isaiah 32:2 sets it in a Millennial context.}
[4] \textcolor[rgb]{0.00,0.00,1.00}{The ungodly are not so: but are like the chaff which the wind driveth away.}\index[AWIP]{The!Psalms!Psa 001:04}\index[AWIP]{ungodly!Psalms!Psa 001:04}\index[AWIP]{are!Psalms!Psa 001:04}\index[AWIP]{not!Psalms!Psa 001:04}\index[AWIP]{so!Psalms!Psa 001:04}\index[AWIP]{but!Psalms!Psa 001:04}\index[AWIP]{are!Psalms!Psa 001:04 (2)}\index[AWIP]{like!Psalms!Psa 001:04}\index[AWIP]{the!Psalms!Psa 001:04}\index[AWIP]{chaff!Psalms!Psa 001:04}\index[AWIP]{which!Psalms!Psa 001:04}\index[AWIP]{the!Psalms!Psa 001:04 (2)}\index[AWIP]{wind!Psalms!Psa 001:04}\index[AWIP]{driveth!Psalms!Psa 001:04}\index[AWIP]{away!Psalms!Psa 001:04}\index[NWIV]{15!Psalms!Psa 001:04}\footnote{The word ``chaff'' is found 14 times in 14 verses in scripture, only twice in the New Testament (Mathew 3:12 and Luke 3:17) speaking of the chaff which will be burnt up with unquenchable fire. See: (1) Job 21:18, (2)  Psalm 1:4 (here),  (3) Psalm 35:5, (4)  Isaiah 5:24, (5)  Isaiah 17:13, (6)  Isaiah 29:5, (7) Isaiah 33:11, (8)  Isaiah 41:15, (9) Jeremiah 23:28, (10) Daniel 2:35, (11) Hosea 13:3, (12) Zephaniah 2:2, (13) Matthew 3:12, and (14) Luke 3:17.} 
[5] \textcolor[rgb]{0.00,0.00,1.00}{Therefore the ungodly shall not stand in the judgment, nor sinners in the congregation of the righteous.}\index[AWIP]{Therefore!Psalms!Psa 001:05}\index[AWIP]{the!Psalms!Psa 001:05}\index[AWIP]{ungodly!Psalms!Psa 001:05}\index[AWIP]{shall!Psalms!Psa 001:05}\index[AWIP]{not!Psalms!Psa 001:05}\index[AWIP]{stand!Psalms!Psa 001:05}\index[AWIP]{in!Psalms!Psa 001:05}\index[AWIP]{the!Psalms!Psa 001:05 (2)}\index[AWIP]{judgment!Psalms!Psa 001:05}\index[AWIP]{nor!Psalms!Psa 001:05}\index[AWIP]{sinners!Psalms!Psa 001:05}\index[AWIP]{in!Psalms!Psa 001:05 (2)}\index[AWIP]{the!Psalms!Psa 001:05 (3)}\index[AWIP]{congregation!Psalms!Psa 001:05}\index[AWIP]{of!Psalms!Psa 001:05}\index[AWIP]{the!Psalms!Psa 001:05 (4)}\index[AWIP]{righteous!Psalms!Psa 001:05}\index[NWIV]{17!Psalms!Psa 001:05}
[6] \textcolor[rgb]{0.00,0.00,1.00}{For the LORD knoweth the way of the righteous: but the way of the ungodly shall perish.}\index[AWIP]{For!Psalms!Psa 001:06}\index[AWIP]{the!Psalms!Psa 001:06}\index[AWIP]{LORD!Psalms!Psa 001:06}\index[AWIP]{knoweth!Psalms!Psa 001:06}\index[AWIP]{the!Psalms!Psa 001:06 (2)}\index[AWIP]{way!Psalms!Psa 001:06}\index[AWIP]{of!Psalms!Psa 001:06}\index[AWIP]{the!Psalms!Psa 001:06 (3)}\index[AWIP]{righteous!Psalms!Psa 001:06}\index[AWIP]{but!Psalms!Psa 001:06}\index[AWIP]{the!Psalms!Psa 001:06 (4)}\index[AWIP]{way!Psalms!Psa 001:06 (2)}\index[AWIP]{of!Psalms!Psa 001:06 (2)}\index[AWIP]{the!Psalms!Psa 001:06 (5)}\index[AWIP]{ungodly!Psalms!Psa 001:06}\index[AWIP]{shall!Psalms!Psa 001:06}\index[AWIP]{perish!Psalms!Psa 001:06}\index[NWIV]{17!Psalms!Psa 001:06}\index[PNIP]{LORD!Psalms!Psa 001:06}\footnote{[RUCKMAN] Here the ungodly man is contrasted with the godly man. His kind are like “chaff,” which is the worthless residue from threshed grain that is tossed up by the winnowing fan and blown away. Three things are true about them \cite{Ruckman1992Psalms}:\begin{compactenum}
\item They shall not be able to ``stand in the judgment,'' (i.e., they will fall down).
\item They will not appear in the ``congregation of the righteous'' (defined in Hebrews 12:22-–24).
\item Their way (and their ways) shall perish (see Psalm 49:12, 17-–20).
\end{compactenum}} \footnote{\textbf{Hebrews 12:22--24} -- But ye are come unto mount Sion, and unto the city of the living God, the heavenly Jerusalem, and to an innumerable company of angels, [23]  To the general assembly and church of the firstborn, which are written in heaven, and to God the Judge of all, and to the spirits of just men made perfect, [24]  And to Jesus the mediator of the new covenant, and to the blood of sprinkling, that speaketh better things than \emph{that of} Abel.} \footnote{\textbf{Psalm 49:12, 17--20} -- Nevertheless man being in honour abideth not: he is like the beasts that perish. [17]  For when he dieth he shall carry nothing away: his glory shall not descend after him.  [18]  Though while he lived he blessed his soul: and men will praise thee, when thou doest well to thyself. [19]  He shall go to the generation of his fathers; they shall never see light.  [20]  Man that is in honour, and understandeth not, is like the beasts that perish.}

