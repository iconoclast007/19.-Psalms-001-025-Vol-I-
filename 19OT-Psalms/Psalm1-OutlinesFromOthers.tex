\subsection{Outlines from Others}

\subsubsection{Bible Study as Worship}
Psalm 1 provides insight into the ungodly man. The word ``ungodly'' is found 27 times in 24 verses in scripture, notably 4 times in Jude 15. Here is used in the corporate sense, as in ungodly people. Usages of the word:\footnote{Kevin Cauley}\\ \\
\index[speaker]{Kevin Cauley!Psalm 001 (Bible Study as Worship}
\index[series]{Psalms (Kevin Cauley)!Psalm 001 (Bible Study as Worship}
\index[date]{2002/07/03!Psalm 001 (Bible Study as Worship) (Kevin Cauley)}
\noindent \textbf{Introduction: }A study of worship in the Bible. In this lesson we will look at Bible Study as Worship. Bible Study is worshiping God because we are 1) Reciting God's Will. 2) Respecting and Honoring God's Word. 3) Retaining God's thoughts in our hearts. 4) Retooling our lives in God's image.  Psalm 1:
\begin{compactenum}[1.][9]
	\item This first Psalm is a Psalm of introduction.
	\item It introduces us to the concept of studying God's word with the intention of applying it to life.
	\item A blessing is pronounced upon the one who does not progress in sin.
	\item He doesn't progress in sin because he has a delight in God's word.
	\item He meditates and thinks about God's word day and night.
	\item The beautiful results are that he is like a tree planted by the waters that brings forth fruit.
	\item This Psalm gives us a picture of worshiping God through the study of His word.
	\item Remember our definition of worship, to bend the knee and prostate oneself toward.
	\item How is this the case with Bible study?
\end{compactenum}
\textbf{\\Discussion: }Discussion: Christians worship God when we study His word because we are
\begin{compactenum}[I.][8]
	\item \textbf{Reciting God's Will} 
	\item  \textbf{Respecting and Honoring God's Words} 
	\item  \textbf{Retaining God's Words in Our Heart} 
	\item  \textbf{Retooling Our Lives in God's Image} 
\end{compactenum}

\subsubsection{Characteristics of the Successful Believer}
Psalm 1 provides insight into the ungodly man. The word ``ungodly'' is found 27 times in 24 verses in scripture, notably 4 times in Jude 15. Here is used in the corporate sense, as in ungodly people. Usages of the word: \\ \\
\index[speaker]{sermonnotebook.org!Psalm 001:01-06 (Characteristics of the Successful Believer}
\index[series]{Psalms (sermonnotebook.org)!Psalm 001 (Characteristics of the Successful Believer}
\index[date]{unknown!Psalm 001 (Characteristics of the Successful Believer) (sermonnotebook.org)}
\noindent \textbf{Introduction: }There are many yardsticks by which men measure success. Among them are: wealth, power, position, prestige, etc. These are all indicators of worldly success. However, measuring your spiritual success and your success as a Christian is a little more difficult. Often, there is little tangible, visible evidence, which you can put your hands on to prove you are living a successful Christian life. Generally, success for the believer is more internal that external. What I mean by that is this, the small amount of external evidence we produce is the result of much internal activity in our hearts and lives. That is why it is difficult for God’s children, who live in a materialistic world, to gauge their success in their walk with God. Thankfully Psalm 1 exists. In this worthy doorkeeper to the psalms, we find insight into what makes a believer successful. Tonight, as the plumbline of God’s Word moves alongside you life, see for yourself just how you measure up to God’s standard for successful Christian living. As we move through this Psalm, allow me to share with you The Characteristics Of The Successful Believer.

\begin{compactenum}[I.][8]
	\item The \textbf{Path} of the Successful Believer \index[scripture]{Psalms!Psa 001:01}(Psalm 1:1)
	\begin{compactenum}[A.]
		\item The Successful believer is separated in his walk of life.
		\item The downward progress – Walk, Stand, Sit. (This is the path Lot took – Gen. 19. It eventually led to his total downfall!)
		\item The successful believer realizes that there is a vast difference between himself and the world he was saved out of, and he lives accordingly!
	\end{compactenum}
	\item The \textbf{Pleasure} of the Successful Believer \index[scripture]{Psalms!Psa 001:02}(Psalm 1:2)
	\item The \textbf{Prosperity} of the Successful Believer \index[scripture]{Psalms!Psa 001:03}(Psalm 1:3)
	\begin{compactenum}[A.]
		\item His \textbf{Position} – By the River! Always close to the life giving resources. (ILLUSTRATION. This was meaningful to Israel with her mostly arid conditions.) The tree planted by the river is never dry and wilted, but is green, lush and lovely. (ILLUSTRATION. The believer who lives close to God will never be dry and wilted either. He will be vibrant, lively and productive.) (ILLUSTRATION. Many never know the joy of drawing off Christ daily! As a result, they are spiritually wilted and dead looking.) The droughts of life and the dry seasons never seem to affect the believer who is planted near the river. He is connected to an unfailing source of life and strength.
		\item His \textbf{Prominence} – ILLUSTRATION. A tree. The life of the successful believerstands heads above all those around him. It is easily seen when a man draws from the Lord. (Ill. Men will know when you have been with Jesus – Acts. 4:13)
		\item His \textbf{Permanence} – Planted – Unlike some plants, which live for a season and die out, this tree, has sunk its root deep and has a hidden source of life. (ILLUSTRATION. The value of private prayer and Bible Study.) (ILLUSTRATION. Planted – literally "transplanted." A tree cannot transplant itself, neither can a man transplant himself into the Kingdom of God. It is wholly a work of God’s grace. And, He always plants us in good soil, near the water supply. However, after we are planted, it is our responsibility to draw from the resources, which God has provided.)
		\item His \textbf{Productivity} – "Brings forth fruit" – The successful believer is a blessing to all those around him, because his fruit is plentiful. (ILLUSTRATION. John 7:38) (ILLUSTRATION. Old apple tree in the cow pasture. Man, cows, birds and insects all benefited from the fruit off this old tree.) (ILLUSTRATION. You may never know just who is feeding off your life!)
		\item His \textbf{Predictability} – "In his season" This tree isn’t a freak. Just as there are seasons of fruit bearing, so there are times of rest and growth. As believers, we aren’t to worry over the fruit. That is the Father’s business! When everything else is as it should be, then the fruit will come in its season – John 15:1-5.
		\item His \textbf{Perpetuity} – "leaf shall not fade" – The successful believer is like an evergreen. He is always surrounded by the green of life. (ILLUSTRATION. The trees in the wintertime. The hardwoods and leafy trees are look dead, but the evergreens stand out as islands of life in a sea of deadness. They are unaffected by winter or weather, but they are always the same.) (ILLUSTRATION. our lives should be lives of consistency! We are called on to be a stable, faithful and dependable people – 1 Cor. 15:58) The successful believer is consistent. The curve balls of life are unable to knock him off course. (ILLUSTRATION. Thank the Lord for consistent people!) (ILLUSTRATION. Life lived by this river in unchanging.)
		\item His \textbf{Prosperity} –"Whatsoever he doeth, it shall prosper" – In other words, God will bless the successful believer. His personal life, his family life, his business life, his church life, his spiritual life will all be blessed of the Lord. That isn’t to say that there won’t be stormy seas, but the successful believer will be able to sail them with Jesus until they are calm once again!
	\end{compactenum}
\end{compactenum}


\subsubsection{How to Identify a Blessed Man}
\index[speaker]{Clark Herring!Psalm 001 (How to Identify a Blessed Man}
\index[series]{Psalms (Clark Herring)!Psalm 001 (How to Identify a Blessed Man}
\index[date]{2018/01/10!Psalm 001 (How to Identify a Blessed Man) (Clark Herring)}
\begin{compactenum}[I.][6]
	\item His Decision \index[scripture]{Psalms!Psa 001:01}(Psalm 1:1)
	\begin{compactenum}[A.]
		\item Sprint (walk)
		\item Stand
		\item Seat
	\end{compactenum}	
	\item His Delight \index[scripture]{Psalms!Psa 001:02}(Psalm 1:2)
	\begin{compactenum}[A.]
		\item Law (2a)
		\item Lord (2b)
		\item Length (2c)
	\end{compactenum}	
	\item His Duration \index[scripture]{Psalms!Psa 001:03}(Psalm 1:3)
	\begin{compactenum}[A.]
		\item Fruitful (3a)
		\item Faithful (3b)
		\item Fortune (3c)
	\end{compactenum}	
	\item His Discouragement \index[scripture]{Psalms!Psa 001:04}(Psalm 1:4)
	\begin{compactenum}[A.]
		\item Satan
		\item Self
		\item Scorners
		\item Saints (jealousy)
	\end{compactenum}	
	\item His Destination \index[scripture]{Psalms!Psa 001:05}(Psalm 1:5)
	\begin{compactenum}[A.]
		\item We have a Place
		\item We have a Position
		\item We have a Pardon
	\end{compactenum}	
	\item His Direction \index[scripture]{Psalms!Psa 001:06}(Psalm 1:6)
	\begin{compactenum}[A.]
		\item Christ (6a)
		\item Confidence (6b)
		\item Comparison (6c)
	\end{compactenum}	
\end{compactenum}

\subsubsection{The Blessed Man}
\index[speaker]{Brian Eades!Psalm 01 (The Blessed Man)}
\index[series]{Psalm (Brian Eades)!Psalm 01 (The Blessed Man)}
\index[date]{2018/11/01!Psalm 01 (The Blessed Man) (Brian Eades)}
\textbf{Source}: Facebook, ``Sermon Hints'' group\\
\textbf{Introduction}: The Bible refers to the word "bless" 463 times; the word "curse" (being the antonym) which is mentioned only 183 times. What does it mean to be blessed? There are people on Earth who are blessed yet have no money or even health. Look at Job...His story is one of tragedy, yet he was the most blessed man on the face of the Earth!\\
\\
BLESSINGS-"Blissfully contented, definely or supremely favored, fortunate, that which brings thankfulness and appreciation."\\
\\
1 Timothy 6:6-10 (KJV) But godliness with contentment is great gain. For we brought nothing into this world, and it is certain we can carry nothing out. And having food and raiment let us be therewith content. But they that will be rich fall into temptation and a snare, and into many foolish and hurtful lusts, which drown men in destruction and perdition. For the love of money is the root of all evil: which while some coveted after, they have erred from the faith, and pierced themselves through with many sorrows.
\begin{compactenum}[I.][7]
	\item The \textbf{INTRODUCTION}   (of the entire Psalms) "Blessed"
	\item The \textbf{INDIVIDUALITY}   "is THE man"
	\item The \textbf{INSTRUCTIONS} 
	\begin{compactenum}[A.]
		\item AVOIDS BAD ADVICE
Walking not in the counsel of the ungodly
(backing)
		\item AVOIDS BAD ACTIONS
Standing not in the way of sinners (behavior)
		\item AVOIDS BAD ATTITUDES
Not sitting in the seat of the scornful (berating)
	\end{compactenum}
\item The \textbf{INSPIRATION} 
	\begin{compactenum}[A.]
		\item READING THE SCRIPTURES (comforted)
		\item ROOTED IN THE STREAM (consistent)
		\item REPRODUCTIVE IN THE SOIL (converting)
		\item READY FOR A SUCCESSFUL LIFE
	\end{compactenum}
\end{compactenum}