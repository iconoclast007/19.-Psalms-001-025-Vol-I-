\subsubsection{The, little ``g,'' gods in scripture}

The word is used 244 times in scripture. Here, we'' provide a description, history, and future of this group of beings.  Genesis 3:5, the first mention, tells us that they were around for the temptation scene with Eve in the Garden.\footnote{\textbf{Genesis 3:5} - For God doth know that in the day ye eat thereof, then your eyes shall be opened, and ye shall be as gods, knowing good and evil.} They are said to be knowledgeable, specifically knowing good and evil. They are beings with a  moral compass. Job 38:7 calls these beings the ``sons of God.''\footnote{\textbf{Job 38:7} -  When the morning stars sang together, and all the sons of God shouted for joy?} This verse is supported by Psalm 82:6, which calls them ``children of the most high.''\footnote{\textbf{Psalm 82:1-8} - God standeth in the congregation of the mighty; he judgeth among the gods. [2] How long will ye judge unjustly, and accept the persons of the wicked? Selah. [3] Defend the poor and fatherless: do justice to the afflicted and needy. [4] Deliver the poor and needy: rid them out of the hand of the wicked. [5] They know not, neither will they understand; they walk on in darkness: all the foundations of the earth are out of course. [6] I have said, Ye are gods; and all of you are children of the most High. [7] But ye shall die like men, and fall like one of the princes. [8] Arise, O God, judge the earth: for thou shalt inherit all nations.}These ``gods'' are commanded to worship the Lord in Psalm 97:7.\footnote{\textbf{Psalm 97:7} - Confounded be all they that serve graven images, that boast themselves of idols: worship him, all ye gods.} Psalm 96:4-5 identify these ``gods'' as idols.\footnote{\textbf{Psalm 96:4-5} - For the LORD is great, and greatly to be praised: he is to be feared above all gods. [5] For all the gods of the nations are idols: but the LORD made the heavens.}\\
\\
\noindent The ``gods'' are no doubt the behinds behind the mythology of every historical culture. They are the beings involved in the sordid saga of Genesis 6 (described in Jude).  Their activities brought to bear such things as the fish-god Dagon in 1 Samuel 5:3-5. Such idolatry is aptly seen in the ``queen of heaven'' in Jeremiah 44:17-19, along with all her aliases. Idolatry is alos known as ``aids to worship!''\\
\\
\noindent Stated by Ruckman, ``doting on a statue or idol summons up a resident spirit. This explains he most primitive forms of religious belief, called `animism': the teaching that different spirits reside in tress, lakes, rivers, mountains, doors, windows, animals, etc.'' This probably explains the plethora of Catholic saints, and the Hindu 33 million gods. ``This idolatry draws demons, and eventually a nation becomes so infested with them that God abandons it.'' \cite{Ruckman1992PsalmsV2}
